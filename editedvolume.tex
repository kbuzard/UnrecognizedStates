\documentclass[12pt,letterpaper, notitlepage]{article}
\usepackage{epsf,graphicx,psfrag}
\usepackage{natbib}
\usepackage{setspace}
\usepackage{fullpage}
\usepackage[reqno]{amsmath}
\usepackage{endnotes}
\usepackage{natbib}
\usepackage{amsfonts}
\usepackage{amssymb,enumerate}
\usepackage{dcolumn}
\usepackage{verbatim}
%\usepackage{hypertex}
\usepackage[bookmarks, colorlinks=true, plainpages = false, citecolor = blue, urlcolor = blue, filecolor = blue, breaklinks]{hyperref}
\usepackage{amssymb}
\usepackage{amsmath}
\usepackage[hmargin=3.0cm,vmargin=3.0cm]{geometry}
%\usepackage{geometry}
\usepackage{nicefrac}
\usepackage{pdfsync}
\usepackage{url}
\usepackage{pst-all}
\usepackage{pdflscape}
\usepackage{rotating}
\usepackage[T1]{fontenc}
\usepackage[sc]{mathpazo}
\usepackage{graphicx,verbatim,scalefnt}

\newcommand{\ta}{\theta}
\newcommand{\Ta}{\Theta}
\newcommand{\ga}{\gamma}
\newcommand{\un}{\underline}
\newcommand{\ov}{\overline}
\linespread{1.05} % Palatino needs more leading (space between lines)
\title{Unrecognized States:  Theory and Cases}

\date{\today}
\begin{document}
\begin{titlepage}
\maketitle
\thispagestyle{empty}
%\center{Word Count: 12,842\\ This count includes title page, abstract, footnotes, and bibliography, but excludes the technical appendix, which does not need to be included in the print version of the article.}

\begin{abstract}
\noindent  Unrecognized states destabilize the international system and impoverish their residents. Thus, unrecognized statehood is a profoundly undesirable outcome, and yet it is often a  stable equilibrium. Game theoretic modeling has proven effective at clarifying the strategic logic that sustains unrecognized states, and offers insight into possible paths to resolution. In this chapter we draw on these insights, and illustrate them with discussion individual cases. The game theory on which we draw analyzes not only the actions of unrecognized states and the home states from which they are attempting to secede, but also the patrons that support these unrecognized states and the actors in the international community who work to induce peaceful settlement. In this piece we focus particularly on evaluating the policy options available to peace and development-seeking actors in the international community as they work to resolve these stalemated conflicts.
\end{abstract}

\end{titlepage}
\setcounter{page}{1}

\doublespacing


\section*{Introduction}

Unrecognized states destabilize the international system and impoverish their residents. Because their territorial borders are contested, the threat of violent conflict is ever present. Facing sanctions, unable to sign trade agreements or receive most foreign aid, their residents are isolated and impoverished. Thus, unrecognized statehood is a profoundly undesirable outcome, and yet it is often a  stable equilibrium. In addition to Somaliland and the Turkish Republic of Northern Cyrpus, many of the unrecognized states that emerged when the Soviet Union collapsed still enjoy \emph{de facto} independence a quarter century later.


1. Unrecognized states are destabilizing to the international system, bad for their residents, and yet often very long in tenure.
2. Game theory helps us understand the mechanisms by which they are sustained, and the ways in which they can be transformed
3. Engagement with the details of actual cases informs us to what actual paths to resolution of these disputes look like, and what roadblocks stand in the way


[Include an updated version of Table 1 from the JLEO article. Ben's RA will update this]


\section*{Self Determination and Foreign Influence}
[Summary of our theory in approx 1500 words]

\section*{How Unrecognized Statehood Ends}

Unrecognized Statehood exists as a halfway point between recognized independent statehood and reunification with the home state. Unrecognized statehood ends when either recognized statehood is achieved, or reunification occurs.  In the following sections we discuss how these transitions occur -- both through negotiated settlement and as a result of decisive military victory by one side or the other.  

\subsection*{Reunification via Military Reconquest}

Most attempted secessions end in military defeat before territorial control is ever achieved (Fazal and Griffith 2008). Unrecognized states are thus a relatively elite set of secessionist movements, those that are unusually militarily powerful relative to the home state. However, even among secessionist movements that succeed in maintaining territorial control for a minimum of two years, the most common form of resolution remains military reconquest by the home state. When unrecognized states return to war with the home state, it is almost always the home state that initiates and the home state that is victorious (we discuss military victory by the secessionists in a later section).

In the case of most prolonged stalemates, a patron provides enough military assistance to the secessionists to make military reconquest by the patron prohibitively costly.  The 11 cases of military reconquest in Table 1 occur in cases with no patron or cases in which the patron withdraws or reduces its support.  

The cases with no patron are fairly straightforward. For example, Chechnya achieved its \emph{de facto} independence immediately after the fall of the Soviet Union when Russia was very weak.   As Russia strengthened, there was no patron support to offset the relative decline in the Chechens' military capabilities. Over time, Russia's military advantage grew and in 1999 the Russian government invaded and reconquered Chechnya.

It is worth exploring, however, the reasons why a patron might support a secessionist group during its initial rebellion and then withdraw support at a later date. Patrons' strategic interests in the unrecognized state vary from patron to patron, and both budget constraints and salience of interest vary over time. For example, domestic political concerns (primarily ethnic solidarity with the secessionists) induced a modest level of Indian support for the Tamil Tigers in Sri Lanka 1983-1987.  These domestic political concerns were eventually outweighed by broader strategic security concerns and a desire for regional stability. In 1987 the Indian government signed a peace accord with Sri Lanka (the home state) and largely withdrew their support from the Tamil secessionists, even sending in peacekeepers that later clashed with the secessionists militarily.\footnote{Singer 1992}

The patron's decision to withdraw support for the secessionists is sometimes motivated by interactions between the patron and the international community, which may bribe or coerce the Patron to abandon its support. In an extreme example involving both sanctions and direct military confrontation, the United States and other members of the North Atlantic Treaty Organization (NATO) coerced Serbia into withdrawing its support from Republika Srpska and Republika Srpska Krajina, both of which had secured \emph{de facto} independence after the collapse of Yugoslavia.\footnote{For an excellent discussion of the case of Republika Srpska, see Zahar 2004.} 

\subsection*{Negotiated Reunification}

Just as military reconquest becomes more likely when Patron support is withdrawn or declines, so too does negotiated reunification. Negotiated agreements are struck when the patron does not contribute sufficiently to prevent the secessionists from preferring ceding to the status quo, and when a deal is available that both sides prefer to war. Since WWII, four peacefully negotiated reunifications have occurred.\footnote{We limit our discussion here to entities that had existed in a period of stalemate prior to reaching a settlement -- i.e. those that had maintained territorial control for at least two years.} Secessionists in Ajara, Bouganville, and Gagauzia have opted to rejoin the home state.  In all four of the cases of negotiated settlement, the observed outcomes seem to match the model well: the payoffs to the secessionist elite from ceding have been low, and the payoffs to the central government high.

In Ajara, where the level of patron (Russian) support was quite low, the choice facing the secessionist elite was between agreeing to reunify with Georgia or facing military defeat. In Bouganville, which separated from Papua New Guinea, secessionists lacked not only a patron, but also a clear preference for secession -- demands for secession had emerged only late in a struggle that began as an effort to stop a mining operation.\footnote{Ghai and Regan 2006}  Here the value of status to the secessionists was actually quite low, and they were willing to surrender it in exchange for relatively small side payments.

However, despite past failures, or theory suggests that a sufficiently motivated patron can induce negotiated settlement if it so chooses. The means through which the international community might induce negotiated settlement are discussed in detail in the section on policy implications. It is notable, however, that we do not expect any future cases of peacefully negotiated independence.  While negotiated reunification is the preferred outcome of the international community, and they may be willing spend to achieve it, recognized statehood is generally not the preferred outcome of the patron or the international community. Our analysis suggests that the most likely path to recognized statehood is, and will remain, military defeat of the home state.

The difficulty of making credible payments in exchange for status is one clearly demonstrated in the civil war literature.\footnote{e.g. Licklider 1995; Walter 1997, 2002; Fearon and Laitin 2007; Doyle and Sambanis 2006} Unrecognized states generally constitute ``sons of the soil" conflicts in which the central government cannot credibly commit to preserving the local demographic and political dominance of the secessionist elite once the disputed territory reverts to central government control.\footnote{Weimer 1978; Fearon 2004}  While the central government might initially grant the secessionist elite a high level of autonomy in exchange for agreeing to reunification, the level of autonomy is likely to decrease over time, perhaps quite quickly.  Reference to the cases of Abkhazia and Gagauzia are informative here.

At the time of secession, ethnic Akbhaz made up a minority of the population of Abkhazia,\footnote{Cornell 2001; Wooleh 2006} but they now [DOUBLE CHECK WHAT THE CURRENT SITUATION IS] control almost all political posts in the \textit{de facto} government of the region. In 2004, the basket of payments offered by the Georgians in exchange for reunification included a provision guaranteeing that ethnic Abkhaz would retain a majority in the regional parliament, even if the return of internally displaced persons (IDPs) once again placed ethnic Abkhaz in a minority demographic position in the region. The promise, however, was not very meaningful.  First, even if the promise were upheld, it would still mean a step back from the total dominance the ethnic Abkhaz currently enjoy in the region.  Second, if Georgian IDPs returned, they may demand and receive a more equitable system of representation.  These concerns are not abstract; this type of reneging has already occurred in cases that did reach settlement.

Gagauzia achieved de facto independence at the time of the Soviet Union's collapse, but agreed to rejoin Moldova in 1994 as an autonomous region.  While Gagauzia was granted substantial autonomy under the Moldovan Law on the Special Legal Status of Gagauzia, when the governor of Gagauzia, Dmitrii Croiter, moved to assert these powers in 1999, the Moldovan government balked.  By 2002, Croiter was forced to resign, effectively deposed by the Moldovan government.  The Moldovan government jailed a number of other Gagauz politicians, and while Gagauz autonomy was enshrined in the Moldovan constitution in 2003, the de facto level of autonomy has been limited by continued by central government meddling in less-than-free regional elections.\footnote{Protsyk (2010) provides an account of the "salami tactics" by which Moldovan authorities have gradually reclaimed powers originally granted to the regional government.} The payoffs to Gagauzia for ceding have turned out to be quite low, and a similar fate can rationally be expected by other unrecognized states who choose to cede.\footnote{Roper (2002) argues that secessionists in Transnistria are wary of negotiated reunification precisely because of the creeping re-centralization they have observed in Gagauzia.}

\subsection*{Recognition via Secessionist Military Victory}
While the path to independent statehood via secession is an extremely narrow one, recognition does sometimes occur. It has occurred primarily in cases where the secessionists (often supported by a patron) are so strong militarily that they not only achieve territorial control in the initial conflict, but also threaten the home state government outside the unrecognized state.  Bangladesh and Eritrea both secured recognition as part of the peace agreement ending the war of secession.  

[DISCUSS BANGLADESH AND ERITREA IN DETAIL]


\section*{Negotiated Recognition}
No unrecognized state has yet managed to gain recognition from the home state when recognition or a referendum was not agreed to as a condition of ending the initial war of secession. Wars that have reignited after a period of unrecognized statehood have always either resulted in reunification or left the status quo intact.  However, if an unrecognized state were to gain an outright military victory over the home state at any time, this does represent a plausible path to recognition. Once unrecognized statehood has emerged as an equilibrium, however, the path to recognition is narrower still.

DISCUSS KOSOVO AND SOUTH SUDAN HERE.


\section*{Policy Implications: Options for The International Community}
 In general, the international community has preferences for reunification over independence, for resolution over the status quo, and for peace instead of war.  In this section we consider  four means through which the international community might pursue these ends: sanctions against the secessionist region, direct incentives provided to the secessionists in exchange for ceding, enforcement of concessions offered by the home state, and direct coercion of the patron.

The intended effect of sanctions is to make the status quo less appealing vis-\`{a}-vis ceding. However any sanctions that increase the secessionists' hostility toward reunification will also increase the range of conditions under which war will be chosen.  Sanctions can have this effect if they reduce the secessionists' quality of life under the status quo %(via imposition of economic costs)
and reduce the quality of the deal secessionists expect to get if they opt for negotiated resettlement. As the peaceful options become worse, war becomes relatively more attractive.
 Compounding this, sanctions that reduce the secessionists' military capabilities (and thus reduce the secessionists' expected payoffs from war) also have the effect of making military reconquest easier for the home state, making it more likely that the home state will attack. In either case, the range of conditions under which war will be initiated becomes broader.\footnote{In most cases, the military position of the home state is stronger than that of the secessionists, so a further tip in the balance of military power toward the home state is more likely to induce war than a similar change in favor of the secessionists.}

%However, if $\delta_s$ is high or the payoffs from war exceed the lowered payoffs from ceding for any other reason (such as animosity of the secessionist public toward the sanctions-imposing home state government), the effect of sanctions will be to induce war rather than negotiated settlement.

There is a better way. If the international community tries to promote settlement by supplementing the payoffs from unification, they are able to induce negotiated settlement without simultaneously increasing the risk of war.  This can be done either through promises of benefits to the unrecognized state provided directly by the international community, like aid, or by a commitment from the international community to serve as a third-party guarantor of side payments promised by the ceding side. In the case of contingent promises of aid, the calculation is relatively straightforward: 1) the promise of aid must be credibly contingent on negotiated settlement, and 2) the aid offered must be valued more highly than the concessions required to reach an agreement. It is the second condition that is most problematic. Because both sides place such a high value on status (independence vs. reunification), even large amounts of aid are likely to be valued less than the concessions necessary to reach an agreement.

Serving as a third-party guarantor of autonomy rights is a way for the international community to potentially overcome problems of indivisibility and commitment and help the parties reach a credible compromise on status\footnote{Walter 2002} However, this strategy is only tenable when the only impediment to settlement is the unenforcability of a bargain, and when the international community is credible as an enforcer of that bargain.

%Credible enforcement of future autonomy rights can be viewed either as increasing the value of available side payments or as making the central issue of contention divisible. In either view, a range of previously untenable agreements are made possible.

In Southern Sudan, the international community invested substantial resources to help negotiate a settlement and to ensure that the Sudanese government government both allowed the promised a referendum and respected its results. While the international community acted in Southern Sudan to enforce independence, not autonomy, it has shown itself capable of enforcing difficult concessions by the home state government. This bodes well for the future credibility of the international community as a third-party enforcer.  However, the role of the international community in enforcing other past agreements might give secessionists pause. For example, a referendum on independence in Western Sahara, which the UN ruled to be necessary more than thirty years ago, has never come to pass.\footnote{For a thorough analysis of the Western Sahara case, see Zunes and Mundy (2010).}  Nonetheless, it is possible for the international community to invest resources to enforce agreements, allowing for negotiated settlements that would otherwise be impossible to reach.

To show that it is possible for the international community to enforce the terms of negotiated agreements at a reasonable cost is not sufficient to imply that such an outcome is likely. The political will necessary to achieve success in Southern Sudan was motivated largely by the magnitude of the atrocities that accompanied the war of secession, and enforcement was made credible, in part, due to the weakness of Sudan relative to the international community. Enforcing the terms of an agreement between Russia and Georgia, for example, would be more difficult.

It is also possible for the international community to affect the payoffs of the patron through interactions in other games outside of our model. Such actions would manifest themselves within the model as reductions in the patron's willingness to pay to sustain the status quo. If the patron is unwilling to pay to sustain the status quo, the war payoffs and status quo payoffs of the secessionists will decline over time, eventually leading to either war or negotiated settlement. Under these conditions, the within-game costs to the international community of inducing negotiated reunification also fall.

In this section we have argued that successful intervention by the international community is possible. The key, however, is motivation: the international community is capable of inducing peaceful settlement when it is willing to invest the resources necessary. However, strong preferences of secessionists against reunification and the opposing intervention of the patron make the costs of such interventions prohibitively high in most cases.  Unrecognized statehood is a stable equilibrium because the international community is unwilling to invest sufficient resources to outspend the patron and induce its preferred outcome.


\section*{Conclusion}


\end{document}