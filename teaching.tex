\documentclass[12pt]{article}

\addtolength{\textwidth}{1.4in}
\addtolength{\oddsidemargin}{-.7in} %left margin
\addtolength{\evensidemargin}{-.7in}
\setlength{\textheight}{8.5in}
\setlength{\topmargin}{0.0in}
\setlength{\headsep}{0.0in}
\setlength{\headheight}{0.0in}
\setlength{\footskip}{.5in}
\renewcommand{\baselinestretch}{1.0}
\setlength{\parindent}{0pt}
\linespread{1.1}

\usepackage[pdftex,
bookmarks=true,
bookmarksnumbered=false,
pdfview=fitH,
bookmarksopen=true]{hyperref}

\usepackage{amssymb, amsmath, amsthm, bm}
\usepackage{graphicx,csquotes,verbatim}
\usepackage[backend=biber,block=space,style=authoryear]{biblatex}
\setlength{\bibitemsep}{\baselineskip}
\usepackage[american]{babel}
%dell laptop
\addbibresource{C:/Users/Kristy/Dropbox/Research/xBibs/tradeagreements.bib}
%\addbibresource{C:/Users/Kristy/Documents/Dropbox/Research/xBibs/tradeagreements.bib}
\renewcommand{\newunitpunct}{,}
\renewbibmacro{in:}{}


\DeclareMathOperator*{\argmax}{arg\,max}
\usepackage{xcolor}
\hbadness=10000

\newtheorem{proposition}{Proposition}
\newcommand{\ve}{\varepsilon}
\newcommand{\ov}{\overline}
\newcommand{\un}{\underline}
\newcommand{\ta}{\theta}
\newcommand{\al}{\alpha}
\newcommand{\Ta}{\Theta}
\newcommand{\expect}{\mathbb{E}}
\newcommand{\Bt}{B(\bm{\tau^a})}
\newcommand{\bta}{\bm{\tau^a}}
\newcommand{\btn}{\bm{\tau^n}}
\newcommand{\btw}{\bm{\tau^{tw}}}
\newcommand{\ga}{\gamma}
\newcommand{\Ga}{\Gamma}
\newcommand{\de}{\delta}

\begin{document}
\begin{center}
  Unrecognized States: A Theory of Self-Determination and Foreign Influence \\
	By Kristy Buzard, Benjamin A.T. Graham and Ben Horne
\end{center}

\section{Unrecognized states in context (i.e. the strategic situation)}
Unrecognized states destabilize the international system and impoverish their residents. Because the borders of these territories are contested, the threat of violent conflict is ever present. Unrecognized states are unable to sign trade agreements or receive most foreign aid, and most face economic sanctions, a combination that leaves their residents isolated and impoverished. Thus, unrecognized statehood is a profoundly undesirable outcome, and yet it is nonetheless often a stable equilibrium. The long-standing unrecognized states of Somaliland and the Turkish Republic of Northern Cyprus, as well as many of the unrecognized states that emerged when the Soviet Union collapsed, still enjoy de facto independence a quarter century later. Unrecognized states are thus quintessential intractable conflicts: difficult but not impossible to resolve (e.g. Crocker, Hampson, and Aall 2004; 2005).\\

Because unrecognized statehood is such a clearly unfavorable outcome for the two primary parties involved – the unrecognized state itself and the home state from which it is attempting to secede – much of the existing work has either treated unrecognized states as an aberration or a temporary phenomenon, or, conversely, as utterly intractable conflicts rooted in the irrationality of one or more actors. We use game theoretic analysis to challenge these understandings. We argue that unrecognized statehood can, in fact, be a stable equilibrium outcome, and that it can be sustained even when all the players involved are perfectly informed and behaving rationally. Game theory provides value in this context by laying bare the mechanisms by which unrecognized statehood is sustained as a stable equilibrium outcome, and thus illuminating the ways in which these intractable conflicts can be peacefully transformed. \\

We analyze not only the actions of unrecognized states and the home states from which they are attempting to secede, but also the patrons that support these unrecognized states and the actors in the international community who work to induce peaceful settlement. By formalizing the incentives of these third-party actors and the actions available to them, we are able to analyze the conditions under which their conduct can lead to peaceful resolution, and when it can lead to war or continued stalemate (Buzard, Graham, and Horne 2017). \\

We define unrecognized states as territories in which a non-state actor controls territory, governs a population, and seeks but does not receive broad recognition as an independent state. The newest unrecognized states were formed when the Soviet Union collapsed in the early 1990s. In the case of more recent cases of militarily successful secession, such as South Sudan, unrecognized statehood has been avoided. This suggests that, if the stalemates sustaining the six current unrecognized states (Abkhazia, Nagorno-Karabakh, Somaliland, South Ossetia, Transnistria, and the Turkish Republic of Northern Cyprus) can be successfully resolved, a world without unrecognized states is possible (Buzard, Graham and Horne 2019).\\



\vskip.3in
\section{3-player version, Status Quo better than Reunification}
Let us consider a first case where the payoffs for the secessionists are higher in the unrecognized state ('Status Quo') than they are if they cede and rejoin the home state. This, for instance, may be the case very early after taking control of the territory before their economic situation has had a chance to deteriorate.

\subsection{Players and Strategy Spaces}
\begin{enumerate}
	\item Home government (G) chooses $S_G \in \left\{SQ,C\right\}$ where SQ means 'Status Quo' of unrecognized statehood and C means 'Cede' the issue of status and recognize the secessionists as an independent state.
	\item Secessionists (S) choose $S_S \in \left\{SQ,C\right\}$, same as for $G$ except when they Cede, they rejoin the home state.
	\item Patron state (P) chooses $p \in [0,\infty)$ to invest in the secessionists status quo payoffs
%	\item International community (C) chooses $c \in [0,\infty)$ to invest in the secessionists payoffs from rejoining the home state
\end{enumerate}


\vskip.3in
\subsection{Timing and Information}
The patron makes its investment. This investment is seen by all other players. Then the home government and secessions move simultaneously.


\vskip.5in
\subsection{Payoffs}
Let, for instance, ($SQ,C$) mean that $G$ plays $SQ$ and $S$ plays $C$. Then 
\begin{itemize}
	\item The payoffs after ($SQ,SQ$) are $-p,3,2+p$
	\item The payoffs after ($SQ,C$) are $-10-p,5,1$
	\item The payoffs after ($C,SQ$) are $3-p,0,5+p$
	\item The payoffs after ($C,C$) are $-p,3,2$
\end{itemize}

\vskip.5in
\subsection{Analysis}
We proceed by backward induction.
\begin{itemize}
	\item In the simultaneous game between $G$ and $S$, $C$ is dominated by $SQ$ for player $G$.
		\begin{itemize}
			\item Therefore the Nash Equilibrium (NE) of the subgame is determined by $p$: 
				\begin{itemize}
					\item If $2+p \geq 1$ (i.e. $p \geq -1$), the NE is ($SQ,SQ$).
					\item If $2+p \leq 1$ (i.e. $p \leq -1$), the NE is ($SQ,C$).
					\item Recall that the smallest value of $p$ that the Patron can choose is $p=0$.
					\item Since $p\leq-1$ is not possible, ($SQ,C$) is not possible.
					\item Therefore ($SQ,SQ$) will be the equilibrium.
				\end{itemize}
		\end{itemize}
	\item The Patron's decision: Since ($SQ,SQ$) will be the equilibrium no matter what, the Patron will get a payoff of $-p$.
		\begin{itemize}
			\item Since the payoff is $-p$, the Patron maximizes its payoffs by choosing the smallest possible $p$, which is $0$. 
		\end{itemize}
\end{itemize}
Therefore the Subgame Perfect Nash Equilibrium is $p=0$, ($SQ,SQ$) regardless of the value of $p$. 

\vskip.5in
\section{3-player version, Reunification better than Status Quo}
Let us now consider a case where the payoffs for the secessionists are \underline{lower} in the unrecognized state ('Status Quo') than they are if they cede and rejoin the home state. We believe this will be the case a few months or years after they take control of the territory and unrecognized status has caused their economic situation to deteriorate.

\subsection{Players and Strategy Spaces}
Same as in Section 1.


\vskip.3in
\subsection{Timing and Information}
Same as in Section 1.

\vskip.5in
\subsection{Payoffs}
\begin{itemize}
	\item The payoffs after ($SQ,SQ$) are $-p,3,\bm{0}+p$
	\item The payoffs after ($SQ,C$) are $-10-p,5,1$
	\item The payoffs after ($C,SQ$) are $3-p,0,5+p$
	\item The payoffs after ($C,C$) are $-p,3,\bm{0}$
\end{itemize}
The only change(s) from Section 1 is that the secessionist's payoff from ($SQ,SQ$) has been reduced, and the payoffs from ($C,C$) along with it.

\vskip.5in
\subsection{Analysis}
We proceed by backward induction.
\begin{itemize}
	\item In the simultaneous game between $G$ and $S$, $C$ is dominated by $SQ$ for player $G$.
		\begin{itemize}
			\item Therefore the Nash Equilibrium (NE) of the subgame is determined by $p$: 
				\begin{itemize}
					\item If $0+p \geq 1$ (i.e. $p \geq 1$), the NE is ($SQ,SQ$).
					\item If $0+p \leq 1$ (i.e. $p \leq 1$), the NE is ($SQ,C$).
				\end{itemize}
		\end{itemize}
	\item The Patron's decision: if ($SQ,SQ$) is the equilibrium in the subgame, the Patron will get a payoff of $-p$. If ($SQ,C$) is the equilibrium in the subgame, the Patron will get a payoff of $-10-p$.
		\begin{itemize}
			\item Since ($SQ,C$) is the worse outcome for the Patron, the patron does not want to invest at all ($p=0$) if ($SQ,C$) is going to be the outcome. 
				\begin{itemize}
					\item In this case, the Patron chooses $p=0$ and the Patron's total payoff is $-10$.
				\end{itemize}
			\item The Patron prefers ($SQ,SQ$), but to make this happen the Patron must make an investment of $p \geq 1$. Because the Patron's payoff is $-p$, the Patron maximizes its payoffs in this case where it induces ($SQ,SQ$) by choosing the smallest $p$, which is $p=1$. 
				\begin{itemize}
					\item In this case, the Patron chooses $p=1$ and the Patron's total payoff is $-1$.
				\end{itemize}
			\item Since $-1 > -10$, the Patron will choose to invest $p=1$.
		\end{itemize}
\end{itemize}
Therefore the Subgame Perfect Nash Equilibrium is $p=1$, and ($SQ,SQ$) as long as $p \geq 1$, ($SQ,C$) if $p \leq 1$.
\begin{itemize}
	\item $p=1$, ($SQ,SQ$) is an equilibrium outcome.
	\item $p=1$, ($SQ,C$) is also an equilibrium outcome. You can rule this one out by specifying that the Secessionists choose $SQ$ whenever they are indifferent.
\end{itemize}


\vskip.5in
\section{4-player version, Reunification better than Status Quo}
Now we look at the full version of the model including the international community, and continue to assume that the payoffs for the secessionists are lower in the unrecognized state ('Status Quo') than they are if they cede and rejoin the home state. 

\subsection{Players and Strategy Spaces}
We add the international community:
\begin{enumerate}
	\item Home government (G) chooses $S_G \in \left\{SQ,C\right\}$ where SQ means 'Status Quo' of unrecognized statehood and C means 'Cede' the issue of status and recognize the secessionists as an independent state.
	\item Secessionists (S) choose $S_S \in \left\{SQ,C\right\}$, same as for $G$ except when they Cede, they rejoin the home state.
	\item Patron state (P) chooses $p \in [0,\infty)$ to invest in the secessionists status quo payoffs
	\item \textbf{International community (C) chooses} $c \in [0,\infty)$ \textbf{ to invest in the secessionists payoffs from rejoining the home state}
\end{enumerate}



\vskip.3in
\subsection{Timing and Information}
The patron makes its investment. This investment is seen by all other players. \textbf{Then the international community makes its investment. This investment is seen by all other players.} Then the home government and secessions move simultaneously.


\vskip.5in
\subsection{Payoffs}
We add payoffs for the international community, and the international community's investment in the secessionists payoffs from ceding.
\begin{itemize}
	\item The payoffs after ($SQ,SQ$) are $-p,\bm{-c},3,0+p$
	\item The payoffs after ($SQ,C$) are $-10-p,\bm{5-c},5,1\bm{+c}$
	\item The payoffs after ($C,SQ$) are $3-p,\bm{-7-c},0,5+p$
	\item The payoffs after ($C,C$) are $-p,\bm{-c},3,0\bm{+c}$
\end{itemize}

\vskip.5in
\subsection{Analysis}
We proceed by backward induction.
\begin{itemize}
	\item In the simultaneous game between $G$ and $S$, $C$ is dominated by $SQ$ for player $G$.
		\begin{itemize}
			\item Therefore the Nash Equilibrium (NE) of the subgame is determined by $p$: 
				\begin{itemize}
					\item If $0+p \geq 1+c$ (i.e. $p \geq 1+c$), the NE is ($SQ,SQ$).
					\item If $0+p \leq 1+c$ (i.e. $p \leq 1+c$), the NE is ($SQ,C$).
				\end{itemize}
		\end{itemize}
	\item Player $C$'s decision: if ($SQ,SQ$) is the equilibrium in the subgame, Player $C$ will get a payoff of $-c$. If ($SQ,C$) is the equilibrium in the subgame, Player $C$ will get a payoff of $5-c$.
		\begin{itemize}
			\item Since ($SQ,SQ$) is the worse outcome for Player $C$, Player $C$ does not want to invest at all ($c=0$) if ($SQ,SQ$) is going to be the outcome. 
				\begin{itemize}
					\item In this case, the Player $C$ chooses $c=0$ and Player $C$'s total payoff is $0$.
				\end{itemize}
			\item Player $C$ prefers ($SQ,C$), but to make this happen Player $C$ must make an investment of the $c \geq p - 1$. Because Player $C$'s payoff is $5-c$, Player $C$ maximizes its payoffs in the case where ($SQ,C$) is the outcome by choosing the smallest $c$, which is $c=p-1$. 
				\begin{itemize}
					\item In this case, Player $C$'s investment of $c=p-1$ makes its total payoff is $5-c = 5 - (p -1) = 6-p$.
				\end{itemize}
			\item If $6 - p \leq 0$ (where $0$ is Player $C$'s payoff from ($SQ,SQ$)), Player $C$ will choose to invest $c = p-1$ and the outcome will be ($SQ,C$).
			\item If $6 - p > 0$ (i.e. $6 > p$), Player $C$ will choose to invest $c = 0$ and the outcome will be ($SQ,SQ$).
		\end{itemize}
	\item The Patron's decision: if ($SQ,SQ$) is the equilibrium in the subgame, the Patron will get a payoff of $-p$. If ($SQ,C$) is the equilibrium in the subgame, the Patron will get a payoff of $-10-p$.
		\begin{itemize}
			\item Since ($SQ,C$) is the worse outcome for the Patron, the patron does not want to invest at all ($p=0$) if ($SQ,C$) is going to be the outcome. 
				\begin{itemize}
					\item In this case, the Patron chooses $p=0$ and the Patron's total payoff is $-10$.
				\end{itemize}
			\item The Patron prefers ($SQ,SQ$), but to make this happen the Patron must make an investment of $p \geq 6$. Because the Patron's payoff is $-p$, the Patron maximizes its payoffs in this case where it induces ($SQ,SQ$) by choosing the smallest $p$ that leads to ($SQ,SQ$), which is $p=6$. 
				\begin{itemize}
					\item In this case, the Patron chooses $p=6$ and the Patron's total payoff is $-6$.
				\end{itemize}
			\item Since $-6 > -10$, the Patron will choose to invest $p=6$.
		\end{itemize}
\end{itemize}
Therefore the Subgame Perfect Nash Equilibrium is $p=6$, $c = 0$ if $p \geq 6$, $c = p-1$ if $p < 6$ and ($SQ,SQ$) as long as $p \geq c+1$, ($SQ,C$) if $p \leq c+ 1$.
\begin{itemize}
	\item $p=6$, $c = 0$, ($SQ,SQ$) is an equilibrium outcome.
\end{itemize}

\section{Works Cited}

Buzard, Kristy, Benjamin AT Graham, and Ben Horne. 2019. Unrecognized States: Theory and Cases and Policy Implications, in Overcoming Intractable Conflicts: New Approaches to Constructive Transformations, Miriam Elman, Catherine Gerard, Galia Golan, and Louis Kriesberg, eds. London: Rowman $\&$ Littlefield.\\

Buzard, Kristy, Benjamin AT Graham, and Ben Horne. 2017. "Unrecognized States: A Theory of Self-Determination and Foreign Influence." The Journal of Law, Economics, and Organization 33 no. 3: 578-611.\\

Crocker, Chester A., Fen Osler Hampson, and Pamela Aall. 2004. Taming intractable conflicts: Mediation in the Hardest Cases. Washington, DC, United States Institute of Peace Press. \\

Crocker, Chester A., Fen Osler Hampson, and Pamela Aall, Ed. 2005. Grasping the Nettle: Analyzing Cases of Intractable Conflicts. Washington, D.C., United States Institute of Peace Press.\\



\end{document}