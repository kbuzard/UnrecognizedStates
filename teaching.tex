\documentclass[12pt]{article}

\addtolength{\textwidth}{1.4in}
\addtolength{\oddsidemargin}{-.7in} %left margin
\addtolength{\evensidemargin}{-.7in}
\setlength{\textheight}{8.5in}
\setlength{\topmargin}{0.0in}
\setlength{\headsep}{0.0in}
\setlength{\headheight}{0.0in}
\setlength{\footskip}{.5in}
\renewcommand{\baselinestretch}{1.0}
\setlength{\parindent}{0pt}
\linespread{1.1}

\usepackage[pdftex,
bookmarks=true,
bookmarksnumbered=false,
pdfview=fitH,
bookmarksopen=true]{hyperref}

\usepackage{amssymb, amsmath, amsthm, bm}
\usepackage{graphicx,csquotes,verbatim}
\usepackage[backend=biber,block=space,style=authoryear]{biblatex}
\setlength{\bibitemsep}{\baselineskip}
\usepackage[american]{babel}
%dell laptop
\addbibresource{C:/Users/Kristy/Dropbox/Research/xBibs/tradeagreements.bib}
%\addbibresource{C:/Users/Kristy/Documents/Dropbox/Research/xBibs/tradeagreements.bib}
\renewcommand{\newunitpunct}{,}
\renewbibmacro{in:}{}


\DeclareMathOperator*{\argmax}{arg\,max}
\usepackage{xcolor}
\hbadness=10000

\newtheorem{proposition}{Proposition}
\newcommand{\ve}{\varepsilon}
\newcommand{\ov}{\overline}
\newcommand{\un}{\underline}
\newcommand{\ta}{\theta}
\newcommand{\al}{\alpha}
\newcommand{\Ta}{\Theta}
\newcommand{\expect}{\mathbb{E}}
\newcommand{\Bt}{B(\bm{\tau^a})}
\newcommand{\bta}{\bm{\tau^a}}
\newcommand{\btn}{\bm{\tau^n}}
\newcommand{\btw}{\bm{\tau^{tw}}}
\newcommand{\ga}{\gamma}
\newcommand{\Ga}{\Gamma}
\newcommand{\de}{\delta}

\begin{document}
\begin{center}
  Unrecognized States: A Theory of Self-Determination and Foreign Influence \\
	By Kristy Buzard, Benjamin A.T. Graham and Ben Horne
\end{center}

\section{Unrecognized states in context (i.e. the strategic situation)}
Unrecognized states destabilize the international system and impoverish their residents. Because the borders of these territories are contested, the threat of violent conflict is ever present. Unrecognized states are unable to sign trade agreements or receive most foreign aid, and most face economic sanctions, a combination that leaves their residents isolated and impoverished. Thus, \textbf{unrecognized statehood is a profoundly undesirable outcome}, and yet it is nonetheless often a stable equilibrium. \textbf{Somaliland and the Turkish Republic of Northern Cyprus are long-standing unrecognized states}, and many of the unrecognized states that emerged when the Soviet Union collapsed still enjoy de facto independence a quarter century later. Unrecognized states are thus quintessential intractable conflicts: difficult but not impossible to resolve (e.g. Crocker, Hampson, and Aall 2004; 2005).\\

Because unrecognized statehood is such a clearly unfavorable outcome for the two primary parties involved – the unrecognized state itself and the home state from which it is attempting to secede – much of the existing work has either treated unrecognized states as an aberration or a temporary phenomenon, or, conversely, as utterly intractable conflicts rooted in the irrationality of one or more actors. We use game theoretic analysis to challenge these understandings. We argue that unrecognized statehood can, in fact, be a stable equilibrium outcome, and that it can be sustained even when all the players involved are perfectly informed and behaving rationally. \textbf{Game theory provides value in this context by laying bare the mechanisms by which unrecognized statehood is sustained as a stable equilibrium outcome, and thus illuminating the ways in which these intractable conflicts can be peacefully transformed.} \\

\textbf{We analyze not only the actions of unrecognized states and the home states from which they are attempting to secede, but also the patrons that support these unrecognized states and the actors in the international community who work to induce peaceful settlement.} By formalizing the incentives of these third-party actors and the actions available to them, \textbf{we are able to analyze the conditions under which their conduct can lead to peaceful resolution, and when it can lead to war or continued stalemate} (Buzard, Graham, and Horne 2017). \\

\textbf{We define unrecognized states as territories in which a non-state actor controls territory, governs a population, and seeks but does not receive broad recognition as an independent state.} In the most recent militarily-successful secession---South Sudan---unrecognized statehood has been avoided. \textbf{This suggests that, if the stalemates sustaining the eight current unrecognized states (Abkhazia, Donetsk, Luhansk, Nagorno-Karabakh, Somaliland, South Ossetia, Transnistria, and the Turkish Republic of Northern Cyprus) can be successfully resolved, a world without unrecognized states is possible (Buzard, Graham and Horne 2019). On the other hand, as the current conflict in Ukraine demonstrates, the ongoing existence of unrecognized states can be profoundly destabilizing (Sommerlad, 2022).} \\



\vskip.3in
\section{3-player version, Status Quo better than Reunification}
Let us consider a first case where the payoffs for the secessionists are higher in the unrecognized state (`Status Quo') than they are if they cede and rejoin the home state. This, for instance, may be the case very early after taking control of the territory before their economic situation has had a chance to deteriorate.

\subsection{Players and Strategy Spaces}
\begin{enumerate}
	\item Home government (G) chooses $S_G \in \left\{Status Quo,Recognize\right\}$ where the `Status Quo' is unrecognized statehood and `Recognize' means the government cedes the issue of status and recognize the secessionists as an independent state.
	\item Secessionists (S) choose $S_S \in \left\{Status Quo,Rejoin\right\}$, where `Status Quo' is remaining as an unrecognized state and `Rejoin' means they cede the issue of status and rejoin the home state.
	\item Patron state (P) chooses $p \in [0,\infty)$ to invest in the secessionists status quo payoffs
%	\item International community (C) chooses $c \in [0,\infty)$ to invest in the secessionists payoffs from rejoining the home state
\end{enumerate}


\subsection{Payoffs}
Let, for instance, ($Status Quo,Rejoin$) mean that $G$ plays $Status Quo$ and $S$ plays $Rejoin$. Then 
\begin{itemize}
	\item The payoffs after ($StatusQuo,StatusQuo$) are $0-p,3,2+p$
		\begin{itemize}
			\item The patron gets neither its favorite or its least favorite outcome. We normalize the patron's payoff from the political outcome to be zero in this case. But the patron pays a cost of whatever amount $p$ it transfers to the secessionists.
			\item Likewise, the government and the secessionists see this outcome as being between their favorite and least favorite (they just have different favorites...). We normalize the status quo to 3 for the government and 2 for the secessionists. However, the secessionists also receive a transfer of $p$ from the patron.
		\end{itemize}
	\item The payoffs after ($StatusQuo,Rejoin$) are $-10-p,5,1$
		\begin{itemize}
			\item The patron's worst outcome is when the secessionists rejoin the home state, so the patron gets a large negative payoff from the political outcome of -10.
			\item The government likes this outcome better than the status quo, so gets 5 instead of 3.
			\item The secessionists dislike this outcome compared to being unrecognized, so get 1 instead of 2.
		\end{itemize}
	\item The payoffs after ($Recognize,StatusQuo$) are $3-p,0,5$
		\begin{itemize}
			\item This is the patron's favorite outcome, but the patron likes this outcome less than it dislikes reunification, so we make the patron's payoff positive but not as large as the magnitude of dislike for reunification of 10.
			\item The government dislikes this outcome compared to having an unrecognized state, so gets 0 instead of 3.
			\item The secessionists like this outcome better than the status quo, so gets 5 instead of 2.
		\end{itemize}
	\item The payoffs after ($Recognize,Rejoin$) are $-p,3,2$
		\begin{itemize}
			\item This is a strange case, because it's not obvious what should happen if both give in at the same time. We assume that, when this happens, nothing changes---that is, we stay in the status quo. But the secessionists don't get the $p$ because they didn't do what the patron wanted them to do.
		\end{itemize}
\end{itemize}


\subsection{Timing and Information}
The patron makes its investment. This investment is seen by all other players. Then the home government and secessionists move simultaneously.\\

We assume all players have complete information about payoffs; that is, all the players know the payoffs associated with each outcome. This assumption could be relaxed if at least one player were difficult for the other players to model (e.g., Vladimir Putin), but in many situations the players will know each other well enough that their information about payoffs will be complete. This is likely to be the case when the same leaders have been in place for a while so that they have interacted repeatedly. We assume the players know each other well for three reasons: (1) it is applicable in many situations; (2) it is simpler; and (3) it serves to highlight the issues on which we are focused. Thus, the only incomplete information in our strategic situation comes from the simultaneous move between the home government and secessionists.


\subsection{Solution Concept}
This game has a dynamic component, as the government and the secessionists move after seeing the patron's choice. Further, in the previous section we argue that the players have complete information about payoffs. Given our assumption that the players know each other well, it is also reasonable to assume that the players will be able to correctly predict each other's behavior. This is particularly true because this is a high-stakes `game,' so all the players have strong incentives to seek out information and think carefully about what the other players will do before choosing the strategy that will maximize their expected payoffs. As this is common knowledge amongst the players and no commitment devices are available to enforce players' promises, the players cannot make either threats or promises that are incredible. \\

These assumptions (the game has an important dynamic component, information about payoffs in complete, and the actors have common knowledge of sequential rationality and can correctly predict each other's behavior) imply that Subgame Perfect Nash Equilibrium (SPNE) is the appropriate solution concept for this strategic situation (Watson p. 45 and p. 189).

\vskip.5in
\subsection{Analysis}
We proceed by backward induction.
\begin{itemize}
	\item In the simultaneous game between $G$ and $S$, $Recognize$ is dominated by $StatusQuo$ for player $G$.
		\begin{itemize}
			\item Therefore the Nash Equilibrium (NE) of the subgame is determined by $p$: 
				\begin{itemize}
					\item For simplicity, we will assume that the secessionists stay in the Status Quo when they're indifferent (i.e., they must be strictly worse off from being in the Status Quo in order to make a change).
					\item If $2+p \geq 1$ (i.e. $p \geq -1$), the NE is ($Status Quo,Status Quo$).
					\item If $2+p < 1$ (i.e. $p < -1$), the NE is ($Status Quo,Rejoin$).
					\item Recall that the smallest value of $p$ that the Patron can choose is $p=0$.
					\item Since $p <-1$ is not possible, ($Status Quo,Rejoin$) is not possible.
					\item Therefore ($Status Quo,Status Quo$) will be the equilibrium.
				\end{itemize}
		\end{itemize}
	\item The Patron's decision: Since ($Status Quo,Status Quo$) will be the equilibrium no matter what, the Patron will get a payoff of $-p$.
		\begin{itemize}
			\item Since the payoff is $-p$, the Patron maximizes its payoffs by choosing the smallest possible $p$, which is $0$. 
		\end{itemize}
\end{itemize}
Therefore the Subgame Perfect Nash Equilibrium is $p=0$, ($Status Quo,Status Quo$) regardless of the value of $p$. 

\vskip.5in
\section{3-player version, Reunification better than Status Quo}
Let us now consider a case where the payoffs for the secessionists are \underline{lower} in the unrecognized state ('Status Quo') than they are if they rejoin the home state. We believe this will be the case a few months or years after they take control of the territory and unrecognized status has caused their economic situation to deteriorate.

\subsection{Players and Strategy Spaces}
Same as in Section 2.


\subsection{Payoffs}
\begin{itemize}
	\item The payoffs after ($Status Quo,Status Quo$) are $-p,3,\bm{0}+p$
	\item The payoffs after ($Status Quo,Rejoin$) are $-10-p,5,1$
	\item The payoffs after ($Rejoin,Status Quo$) are $3-p,0,5$
	\item The payoffs after ($Rejoin,Rejoin$) are $-p,3,\bm{0}$
\end{itemize}
The only change(s) from Section 1 is that the secessionist's payoff from ($Status Quo,Status Quo$) has been reduced, and the payoffs from ($Rejoin,Rejoin$) along with it.

\subsection{Timing and Information}
Same as in Section 2.

\subsection{Solution Concept}
Same as in Section 2.

\vskip.2in
\subsection{Analysis}
We proceed by backward induction.
\begin{itemize}
	\item In the simultaneous game between $G$ and $S$, $Recognize$ is dominated by $Status Quo$ for player $G$.
		\begin{itemize}
			\item Therefore the Nash Equilibrium (NE) of the subgame is determined by $p$: 
				\begin{itemize}
					\item Again, assume that the secessionists stay in the Status Quo when they're indifferent.
					\item If $0+p \geq 1$ (i.e. $p \geq 1$), the NE is ($Status Quo,Status Quo$).
					\item If $0+p < 1$ (i.e. $p < 1$), the NE is ($Status Quo,Rejoin$).
				\end{itemize}
		\end{itemize}
	\item The Patron's decision: if ($Status Quo,Status Quo$) is the equilibrium in the subgame, the Patron will get a payoff of $-p$. If ($Status Quo,Rejoin$) is the equilibrium in the subgame, the Patron will get a payoff of $-10-p$.
		\begin{itemize}
			\item Since ($Status Quo,Rejoin$) is the worse outcome for the Patron, the patron does not want to invest at all ($p=0$) if ($Status Quo,Rejoin$) is going to be the outcome. 
				\begin{itemize}
					\item In this case, the Patron chooses $p=0$ and the Patron's total payoff is $-10$.
				\end{itemize}
			\item The Patron prefers ($Status Quo,Status Quo$), but to make this happen the Patron must make an investment of $p \geq 1$. Because the Patron's payoff is $-p$, the Patron maximizes its payoffs in this case where it induces ($Status Quo,Status Quo$) by choosing the smallest $p$, which is $p=1$. 
				\begin{itemize}
					\item In this case, the Patron chooses $p=1$ and the Patron's total payoff is $-1$.
				\end{itemize}
			\item Since $-1 > -10$, the Patron will choose to invest $p=1$.
		\end{itemize}
\end{itemize}
Therefore the Subgame Perfect Nash Equilibrium is $p=1$, and ($Status Quo,Status Quo$) as long as $p \geq 1$, ($Status Quo,Rejoin$) if $p < 1$.
\begin{itemize}
	\item $p=1$, ($Status Quo,Status Quo$) is the equilibrium outcome.
\end{itemize}


\vskip.5in
\section{4-player version, Reunification better than Status Quo}
Now we look at the full version of the model including the international community, and continue to assume that the payoffs for the secessionists are lower in the unrecognized state (`Status Quo') than they are if they cede and rejoin the home state. 

\subsection{Players and Strategy Spaces}
We add the international community:
\begin{enumerate}
	\item Home government (G) chooses $S_G \in \left\{Status Quo,Recognize\right\}$ where Status Quo means staying in unrecognized statehood and `Recognize' means to cede the issue of status and recognize the secessionists as an independent state.
	\item Secessionists (S) choose $S_S \in \left\{Status Quo,Rejoin\right\}$, where `Status Quo' is remaining as an unrecognized state and `Rejoin' means they cede the issue of status and rejoin the home state.
	\item Patron state (P) chooses $p \in [0,\infty)$ to invest in the secessionists status quo payoffs
	\item \textbf{International community (C) chooses} $c \in [0,\infty)$ \textbf{ to invest in the secessionists payoffs from rejoining the home state}
\end{enumerate}


\subsection{Payoffs}
We add payoffs for the international community in the second position in the payoff vector since the international community moves second. We also add the international community's investment in the secessionists payoffs from ceding. The international community's payoff from the status quo is normalized to 0. The international community most prefers reunification (the secessionists rejoin, +5) and is strongly opposed to the secessionists being recognized (-7). Here, note that the international community likes its preferred outcome MORE than the patron does, but it dislikes its least preferred outcome LESS than the patron does. This last comparison is crucial for obtaining the outcome we get.
\begin{itemize}
	\item The payoffs after ($Status Quo,Status Quo$) are $-p,\bm{-c},3,0+p$
	\item The payoffs after ($Status Quo,Rejoin$) are $-10-p,\bm{5-c},5,1\bm{+c}$
	\item The payoffs after ($Recognize,Status Quo$) are $3-p,\bm{-7-c},0,5$
	\item The payoffs after ($Recognize,Rejoin$) are $-p,\bm{-c},3,0$
\end{itemize}


\subsection{Timing and Information}
The patron makes its investment. This investment is seen by all other players. \textbf{Then the international community makes its investment. This investment is seen by all other players.} Then the home government and secessions move simultaneously.

\subsection{Solution Concept}
The addition of the international community does not change any of our assumptions related to the solution concept as long as the international community knows the other three players well and vice versa. Thus we continue to use Subgame Perfect Nash Equilibrium as our solution concept.

\vskip.5in
\subsection{Analysis}
We proceed by backward induction.
\begin{itemize}
	\item In the simultaneous game between $G$ and $S$, $Recognize$ is dominated by $Status Quo$ for player $G$.
		\begin{itemize}
			\item Therefore the Nash Equilibrium (NE) of the subgame is determined by $p$ and $c$: 
				\begin{itemize}
					\item If $0+p \geq 1+c$ (i.e. $p \geq 1+c$), the NE is ($Status Quo,Status Quo$).
					\item If $0+p < 1+c$ (i.e. $p < 1+c$), the NE is ($Status Quo,Rejoin$).
				\end{itemize}
		\end{itemize}
	\item Player $C$'s decision: if ($Status Quo,Status Quo$) is the equilibrium in the subgame, Player $C$ will get a payoff of $-c$. If ($Status Quo,Rejoin$) is the equilibrium in the subgame, Player $C$ will get a payoff of $5-c$.
		\begin{itemize}
			\item Since ($Status Quo,Status Quo$) is worse for Player $C$, Player $C$ does not want to invest at all ($c=0$) if ($Status Quo,Status Quo$) is going to be the outcome. 
				\begin{itemize}
					\item In this case, the Player $C$ chooses $c=0$ and Player $C$'s total payoff is $0$.
				\end{itemize}
			\item Player $C$ prefers ($Status Quo,Rejoin$), but to make this happen Player $C$ must make an investment of the $c \geq p - 1$. Because Player $C$'s payoff is $5-c$, Player $C$ maximizes its payoffs in the case where the outcome is ($Status Quo,Rejoin$) by choosing the smallest $c$, which is $c=p-1$. 
				\begin{itemize}
					\item In this case, Player $C$'s investment of $c=p-1$ makes its total payoff is $5-c = 5 - (p -1) = 6-p$.
				\end{itemize}
			\item If $6 - p \leq 0$ (where $0$ is Player $C$'s payoff from ($Status Quo,Status Quo$)), Player $C$ will choose to invest $c = p-1$ and the outcome will be ($Status Quo,Rejoin$).
			\item If $6 - p > 0$ (i.e. $6 > p$), Player $C$ will choose to invest $c = 0$ and the outcome will be ($Status Quo,Status Quo$).
		\end{itemize}
	\item The Patron's decision: if ($Status Quo,Status Quo$) is the equilibrium in the subgame, the Patron will get a payoff of $-p$. If ($Status Quo,Rejoin$) is the equilibrium in the subgame, the Patron will get a payoff of $-10-p$.
		\begin{itemize}
			\item Since ($Status Quo,Rejoin$) is worse for the Patron, the patron does not want to invest at all ($p=0$) if ($Status Quo,Rejoin$) is going to be the outcome. 
				\begin{itemize}
					\item In this case, the Patron chooses $p=0$ and the Patron's total payoff is $-10$.
				\end{itemize}
			\item The Patron prefers ($Status Quo,Status Quo$), but to make this happen the Patron must make an investment of $p \geq 6$. Because the Patron's payoff is $-p$, the Patron maximizes its payoffs in this case where it induces ($Status Quo,Status Quo$) by choosing the smallest $p$ that leads to ($Status Quo,Status Quo$), which is $p=6$. 
				\begin{itemize}
					\item In this case, the Patron chooses $p=6$ and the Patron's total payoff is $-6$.
				\end{itemize}
			\item Since $-6 > -10$, the Patron will choose to invest $p=6$.
		\end{itemize}
\end{itemize}
Therefore the Subgame Perfect Nash Equilibrium is $p=6$, $c = 0$ if $p \geq 6$, $c = p-1$ if $p < 6$ and ($Status Quo,Status Quo$) as long as $p \geq c+1$, ($Status Quo,Rejoin$) if $p \leq c+ 1$.
\begin{itemize}
	\item $p=6$, $c = 0$, ($Status Quo,Status Quo$) is the equilibrium outcome.
\end{itemize}

\section{Conclusion}
We see from the comparison between the outcomes in Sections 2 and 3 that the patron only gets involved if that involvement will change the outcome--if its investments make a difference for the secessionists choice. The patron will pay enough to maintain the status quo as long as the payment is small enough (recall the cost has to be less than 10 in the analysis in Section 3.4). \\

This logic carries over to the version of the game that includes the international community. The patron is willing to pay up to 10 (the difference between its payoffs from the political status in the status quo and rejoin outcomes). As long as the international community is willing to invest less to avoid its least preferred outcome than the patron is willing to invest to avoid its least preferred outcome, the patron will invest, the international community will not invest, and the status quo will prevail.

\section{Works Cited}

Buzard, Kristy, Benjamin AT Graham, and Ben Horne. 2019. Unrecognized States: Theory and Cases and Policy Implications, in Overcoming Intractable Conflicts: New Approaches to Constructive Transformations, Miriam Elman, Catherine Gerard, Galia Golan, and Louis Kriesberg, eds. London: Rowman $\&$ Littlefield.\\

Buzard, Kristy, Benjamin AT Graham, and Ben Horne. 2017. "Unrecognized States: A Theory of Self-Determination and Foreign Influence." The Journal of Law, Economics, and Organization 33 no. 3: 578-611.\\

Crocker, Chester A., Fen Osler Hampson, and Pamela Aall. 2004. Taming intractable conflicts: Mediation in the Hardest Cases. Washington, DC, United States Institute of Peace Press. \\

Crocker, Chester A., Fen Osler Hampson, and Pamela Aall, Ed. 2005. Grasping the Nettle: Analyzing Cases of Intractable Conflicts. Washington, D.C., United States Institute of Peace Press.\\

Sommerlad, J. 2022. ``Ukraine crisis: What are the Donetsk and Luhansk People’s Republics?.'' The Independent. February 25, 2022. https://www.independent.co.uk/news/world/europe/ukraine-donetsk-luhansk-region-russia-war-b2023842.html. Accessed February 26, 2022.



\end{document}