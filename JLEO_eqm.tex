\documentclass[12pt]{article}

\addtolength{\textwidth}{1.4in}
\addtolength{\oddsidemargin}{-.7in} %left margin
\addtolength{\evensidemargin}{-.7in}
\setlength{\textheight}{8.5in}
\setlength{\topmargin}{0.0in}
\setlength{\headsep}{0.0in}
\setlength{\headheight}{0.0in}
\setlength{\footskip}{.5in}
\renewcommand{\baselinestretch}{1.0}
\setlength{\parindent}{0pt}
\linespread{1.1}

\usepackage{amssymb, amsmath, amsthm, bm}
\usepackage{graphicx,csquotes,verbatim}
\usepackage[backend=biber,block=space,style=authoryear]{biblatex}
\setlength{\bibitemsep}{\baselineskip}
\usepackage[american]{babel}
%dell laptop
\addbibresource{C:/Users/Kristy/Dropbox/Research/xBibs/tradeagreements.bib}
%\addbibresource{C:/Users/Kristy/Documents/Dropbox/Research/xBibs/tradeagreements.bib}
\renewcommand{\newunitpunct}{,}
\renewbibmacro{in:}{}


\DeclareMathOperator*{\argmax}{arg\,max}
\usepackage{xcolor}
\hbadness=10000

\newcommand{\ve}{\varepsilon}
\newcommand{\ov}{\overline}
\newcommand{\un}{\underline}
\newcommand{\ta}{\theta}
\newcommand{\al}{\alpha}
\newcommand{\Ta}{\Theta}
\newcommand{\expect}{\mathbb{E}}
\newcommand{\de}{\delta}

\begin{document}
\begin{center}
Unrecognized States -- JLEO Streamlined Model and Equilibrium
\end{center}


\vskip.3in
Kristy
\begin{itemize}
	\item Take indecisive war to decisive war
	\item Change sign on payoff functions for $c$ and $p$
	\item Homogenize discount factors
	\item Decide on Reviewer 1's suggestion to recast Proposition 1 as a ``class of games'' instead of ``sufficient conditions''
		\begin{itemize}
			\item ``I understand why the set of sufficient conditions in this proposition is the key result. However, these mechanisms are essentially just assumptions on payoffs; and therefore I would be inclined to interpret these conditions as characterizations of “a class” of games, where this class would satisfy all the assumptions on payoffs (instead of saying sufficient conditions). \\
				I would recommend clearing up Section 3.1 in terms of packaging the results: 
					\begin{enumerate}
						\item First, I would identify a class of games that satisfy the assumptions on payoffs (sufficient conditions (1) through (6) in Proposition 1). Denote this class as G for example. 
						\item Then, formally define the “status quo” equilibrium – an equilibrium in which the equilibrium outcome is perpetual unrecognized statehood with all the actors’ strategies specified. 
						\item Finally, I would state a much clearer and simpler proposition, e.g., For any game G, there exists a status quo equilibrium. ... Then subsection 3.2 would basically be a discussion of this class of games. 
					\end{enumerate}
		\end{itemize}
	\item Clearly specify punishments / equilibrium path to satisfy editor
		\begin{itemize}
			\item ``Clear and detailed description of the noncooperative game''
			\item ``I would also like to see the key logic laid out better in the text, and the full
equilibrium strategy profiles noted so that it is clear how the players are rewarding and punishing each other on and off the equilibrium path''
		\end{itemize}
	\item Make sure equilibrium concept is clearly stated and defined
		\begin{itemize}
			\item ``it is important to provide formal definitions and to clearly describe the equilibrium concept being utilized. It seems to me that, as Referee 2 states, you are using the Markov perfect equilibrium concept.''
		\end{itemize}
\end{itemize}

\newpage
\begin{quotation}
\noindent \bf \emph{Proposition 1:} \rm  \emph{There exists an equilibrium in which the outcome is perpetual unrecognized statehood. The actions in this equilibrium are for $p$ to invest to create a buffer of $\frac{\beta}{1-\delta_c}$ between the payoffs from ceding and the status quo in the first period and to invest up to $\mu$ each period thereafter to maintain the buffer; for $c$ to pay nothing; and for both $g$ and $s$ to play $Status$ $Quo$ each period.}

\emph{The following are sufficient conditions for such an equilibrium: }

\begin{enumerate}
\item \textit{For both players $g$ and $s$, $Q_{in} \geq L_{in} \ \forall n$: in each period, remaining in the status quo is better than ceding.}

\item \emph{For both players $g$ and $s$, $\frac {Q_{in}}{1-\delta_i} \geq  -\zeta_i+\frac{L_{in}(p_2) +W_{in}(p_1)+Q_{in}(1-p_1-p_2)}{1-\delta_i} \ \forall n$: in each period, the expected outcome under war is worse than the status quo.}

\item \textit{$\frac{-\alpha}{1-\de_p} \geq \frac {\beta}{1-\delta_c}$: reunification is more important for the patron to avoid than for the international community to achieve.}

\item  \textit{$\frac {-\nu\cdot p_{1s}}{1-\delta_c} \geq \frac{\lambda \cdot p_{1s} + \mu}{1 -\de_p} + \frac{\beta}{1 -\de_c}$: recognition of the secessionist state is more important for the international community to avoid than for the patron to achieve.}

\item  \textit{$B_{p1} \geq\frac{\beta}{1-\delta_c} - \left(q_{s1} - l_{s1} \right)$: the patron can afford to deter player $c$ from inducing reunification at the beginning of the game.}

\item \textit{$B_{pn} \geq \mu, \ \forall n>1$: the patron can afford to pay to maintain the status quo.}\footnote{Depending on parameters, condition (5) is more likely to be binding than condition (6). If there is great variance in budget between periods for the patron, such as a greatly reduced budget in some period $n> 1$ compared to period $1$, (6) could be binding.}

\end{enumerate}

\end{quotation}

These conditions for the status quo are sufficient but not necessary. For example, if the status quo initially has a much higher long term payoff than the next best alternative for the secessionists, Condition (6) need not be met to maintain the status quo in the short run. The inequality \emph{will} bind for some set of periods $n \geq 1$ because the secessionist payoffs from the status quo decrease over time. In cases where condition (6) is not met, we can have unrecognized statehood for some time, but it is not a long-run equilibrium outcome.

There are many other potential equilibrium outcomes of this game, including, under the right parameters, immediate ceding by either party as well as fighting (see Section~\ref{sec:alt}). As we are interested particularly in the outcome of long-term unrecognized statehood, here we focus on the question of the existence of an equilibrium that leads to this outcome in perpetuity. We will use the solution concept of subgame perfect equilibrium to show that, given the sufficient conditions in Proposition 1, at least one status quo equilibrium will exist.\\

\noindent {\bf Proof of Proposition 1} \\
Recall that each period is composed of three action stages: player $p$'s investment decision (stage 1), player $c$'s investment decision (stage 2), and the simultaneous game between players $g$ and $s$ (stage 3). Although period $n$ may be reached because in period $n-1$ fighting resulted in a stalemate, both players ceded, or the status quo had been maintained, the strategic landscape in period $n$ is the same. Thus all periods in which the players are able to move are in an equivalence class in which the only possible strategic difference is in the value of the state variables.

Analysis of the posited equilibrium proceeds most naturally by backward induction within each period. Thus we begin with the stage game between the government and secessionists.

\begin{quotation}
\noindent \textit{{\bf Lemma 1:} Conditions 1 and 2 are sufficient for status quo to be the outcome of Stage 3 in any period.} 
\end{quotation}

The proofs of the lemmas are in the Appendix.

Lemma 1 establishes ranges for the payoffs for players $g$ and $s$---gross of investments by the outside players---in which the status quo outcome can occur. The rest of this section tackles the more difficult task of determining the incentives and actions of the outside actors---that is, the patron state and the international community---to impact those payoffs.

To begin, note that either outside actor could invest toward increasing any of the six state variables in a period: $q_{sn},q_{gn},w_{sn},w_{gn},l_{sn},$ and $l_{gn}$.\footnote{In the case of $w_{sn}$ and $w_{gn}$, it is more intuitive to imagine investments increasing the probability of winning a war; because the lottery is additively separable and the ``Win'' outcome is always preferred when it occurs independently in the stage game, this more convenient modeling choice is inconsequential.} It is not necessary to focus on which investment vehicle is optimal for a government to choose; what is essential is to determine which outcomes---i.e. \emph{Recognition, Status Quo}, or \emph{Reunification}---they will target given that they will choose the most efficient way to alter the payoffs of players $s$ and $g$.

%Given their preference orderings, some of these can be ruled out. Player $p$ will not invest to increase $s$'s payoffs from ceding ($l_{sn}$) or $g$'s payoffs from war when it could do so for $s$. It will also not invest in $q_{gn}$ when it can instead increase $l_{gn}$, leaving the list of rational investment choices for $p$ as $q_{sn},l_{gn}$, and $w_{sn}$. Symmetrically, $c$ will not invest in $q_{sn},L_{gn}$, or $w_{sn}$. Given the assumption that the international community follows a norm of encouraging peaceful resolution, it also will not invest in $w_{gn}$, leaving only $l_{sn}$ and $q_{gn}$ as rational investment options for player $c$.

Lemma 2 addresses potential efforts by the international community to influence the outcome toward \emph{Reunification}:

\begin{quotation}
\noindent \textit{{\bf Lemma 2:} When Condition 3 holds, the patron's willingness to invest to maintain the status quo is sufficient to deter the international community from intervening to encourage reunification.}
\end{quotation}

\noindent Condition 3 provides a bound on the amount that player $p$ must be \emph{willing} to invest each period in order to prevent player $c$ from contesting the status quo outcome. To complete the equilibrium construction, we must determine the utility maximizing investments by $p$ and conditions to ensure that it is \emph{able} to make those investments.\footnote{Efforts by the patron to create the conditions for the \emph{Recognition} outcome are addressed in Lemma 3 once the equilibrium investments are established.}

Consider period 1. When Condition (4) holds, the patron will want to invest just enough to create a buffer of $\frac{\beta}{1-\delta_c}$ between $Q_{s1} \ (= q_{s1} + R_{s1})$ and $l_{s1}$ so that $c$ will not invest. That is, in equilibrium, $R_{p1} = \frac{\beta}{1-\delta_c} -(q_{s1} - l_{s1})$ as long as $p$ can afford to make this investment. Thus we need Condition (5): $B_{p1} \geq\frac{\beta}{1-\delta_c} - \left(q_{s1} - l_{s1} \right)$, where $B_{p1}$ is the amount $p$ has available to spend on the conflict in period $1$.

In periods $n > 1$, there are two cases to consider. Either (a) the buffer created in period $n-1$ was precisely the necessary $\frac{\beta}{1-\delta_c}$, or (b) the buffer is larger than $\frac{\beta}{1-\delta_c}$. In case (a), the patron must spend exactly $\mu$ to offset the degradation in the status quo payoffs and re-establish the buffer of $\frac{\beta}{1-\delta_c}$.\footnote{A third case in which the buffer is smaller than $\frac{\beta}{1-\delta_c}$ provides lower welfare to player $p$. Conditions (3) and (5) ensure that player $p$ can avoid this case.} In case (b), the patron can spend less than $\mu$ in period $n$. However, because in each period $q_{sn}$ degrades by $\mu$, eventually the buffer will be reached and we will be returned to case (a). Again, assuming Condition (3) holds, the patron will want to make this investment if its budget allows, and so a sufficient condition is that $p$'s budget is at least as large as $\mu$ in every period $n > 1$(Condition 6). 

With the equilibrium status quo investments determined, we can proceed to Lemma 3, which rules out spending for or against recognition of the secessionist state:

\begin{quotation}
\noindent \textit{{\bf Lemma 3:} When Condition 4 holds, the international community's willingness to invest to avoid recognition of the secessionist state is sufficient to deter the patron from investing to achieve recognition.}
\end{quotation}

In the equilibrium under consideration here, the patron will invest to maintain an outcome that is not its most preferred, but it will not invest to achieve its most preferred outcome. This behavior may appear counterintuitive, but we frequently observe patron states whose preferred outcomes are recognized independence for the secessionists who  nonetheless contribute resources to sustain a status-quo outcome that is costly to all involved. The patron does not attempt to contribute sufficient resources to force recognition by the home state because doing so would induce offsetting expenditures by the international community to prevent this outcome.\footnote{We do not provide conditions on the budget of player $c$ similar to Conditions (5) and (6): since the size of the international community relative to any particular country is large, it can be assumed that a budget constraint does not bind.}

One last possibility is ruled out by Lemma 4: that the patron would invest to encourage the secessionists to fight. 

\begin{quotation}
\noindent \textit{{\bf Lemma 4:} Conditions 3 and 4 ensure that the international community's willingness to invest to discourage new conflict is sufficient to deter the patron from investing to instigate such fighting.}
\end{quotation}

Note that this result depends on an implicit assumption that the patron is not able to skew the odds of the secessionists winning the conflict in a way that cannot be nullified by the international community. All other conflict scenarios are ruled out by the international community's assumed preference to avoid conflict.

Thus, given Conditions (1)-(6) hold, equilibrium strategies for each period in this status quo equilibrium are:
\begin{itemize}
	\item Player $p$ spends more than $c$ is willing to invest to ensure the incentives for the status quo outcome are in place in period $n$ whenever this is affordable. That is, $p$ invests $R_{pn} = \max\left\{0,\frac{\beta}{1-\delta_c} -(q_{sn} - l_{sn})\right\}\leq B_{pn}$. If this inequality is violated, $p$ invests nothing.
	\item Player $c$ invests $R_{cn} = \frac{\beta}{1-\delta_c} -(Q_{sn} - l_{sn})$ if $\frac{\beta}{1-\delta_c} > (Q_{sn} - l_{sn})$. Otherwise, it invests nothing. 
	\item Players $s$ and $g$ play Status Quo as long as it yields the highest continuation value, and play Cede or Fight if continuation values from either exceeds the status quo continuation value. 
\end{itemize}

Equilibrium actions are for $p$ to maintain the status quo by investing $\mu$ each period once the difference in payoffs to $s$ from playing Status Quo and Cede reaches $\frac{\beta}{1-\de_c}$ (with a possible lump sum investment at $n=1$ of up to that amount); for $c$ to not invest and for both $g$ and $s$ to play Status Quo each period. \hfill $\blacksquare$

%Our status quo equilibrium requires that $p$ invests in every period enough to maintain $F^*$ by offsetting the $\mu$ decline in the secessionists' status quo payoffs. The total per-period equilibrium investment $R_{pn}$, a flow payment, in the long run is thus $\mu$ per period in this steady state. In a status quo equilibrium $c$ need not invest at all since stage game payoffs of $g$ do not deteriorate.\footnote{ In the absence of the assumption in the model setup above that player $c$ does not want war, the patron would need to retain a second buffer against $c$ funding war.  The expected value of war for the home state would need to be maintained at a level lower than the status quo payoff by a buffer of $F*=\frac{\beta}{1-\delta_c}$. If this buffer did not exist in the first round of the game, the patron would fund the alteration of payoffs to create the buffer, thus assuring payments by $c$ would be ineffective at trying to change the home state's payoffs to make war more attractive than the status quo.}

By contributing $\mu$ in each period, the patron supplies sufficient resources to the unrecognized state to ensure that the secessionist elite prefers the status quo to surrendering independence. 

\newpage
Appendix

{\bf Proof of Lemma 1} \\
Recall that the upper case notation represents payoffs gross of investments by the outside players. Without loss of generality, consider the incentives to deviate for player $s$ in period $n$ given that the other three players play their equilibrium actions. Given the stationary equilibrium actions, playing Status Quo will lead to $Q_{sn} \ \forall n$, so the continuation value is $\frac{Q_{sn}}{(1-\delta_s)}$. Player $s$'s continuation value from the one-shot deviation to Cede, which leads to an absorbing state with payoff $L_{sn}$ in each period, is $\frac{L_{sn}}{(1-\delta_i)}$. The one-shot deviation to fight results in the war lottery, with a cost of $-\zeta_s$, $W_s$ forever with probability $p_1$, $L_s$ forever with probability $p_2$, and the status quo payoff forever with probability $1-p_1-p_2$. Thus the continuation value is $\zeta_s +\frac{W_{sn}(p_1)+ L_{sn}(p_2) +Q_{sn}(1-p_1-p_2)}{1-\delta_i}$.

In order for player $s$ to play Status Quo, it must be that the continuation value from Status Quo is higher than both that from playing Cede (Condition 1) and Fight (Condition 2). The argument for player $g$ is symmetric. \hfill $\blacksquare$
\\
\\
\noindent {\bf Proof of Lemma 2} \\
Again, in the status quo, we have $U_{pn}^{SQ}= -R_p^{SQ}$ and $U_{cn}^{SQ}= -R_c^{SQ}$. If the outcome is reunification, payoffs in this absorbing state in the period in which the investment is made are $U_{pn}^{RU}= \alpha -R_p^{RU}$ and $U_{cn}^{RU}= \beta -R_c^{RU}$ where $SQ$ and $RU$ distinguish investments under the status quo and reunification scenarios respectively.

Thus the difference in continuation values from switching from the status quo to reunification for player $p$ is $\frac{\alpha + R_p^{SQ} 
}{1 -\de_p}-R_p^{RU}$. For player $c$ it is $\frac{\beta + R_c^{SQ}}{1 -\de_c} -R_c^{RU}$. Neither player will be willing to invest in its least preferred state. That is, $R_p^{RU} = R_c^{REC} =0$. So $c$ is willing to invest up to $\frac{\beta+ R_c^{SQ}}{1 -\de_c}$ toward reunification---or preventing the status quo---and $p$ is willing to invest $\frac{-\left(\alpha + R_p^{SQ} 
\right)}{1 -\de_p}$ each period toward maintaining it.

Player $c$ would be able to deter player $p$ from investing to maintain the status quo---e.g. counter an addition to $q_{sn}$ with an investment to $l_{sn}$---as long as $R_c^{RU} = \frac{\beta + R_c^{SQ}}{1 -\de_c} \geq Q_{sn} - l_{sn}$, where $Q_{sn}$ is gross of any investment by $p$.\footnote{Here we have assumed that the payoff relationship at $n$ is such that ceding is better than fighting. If this relationship is reversed, similar analysis and results hold.} That is, the required investment must be no more than the gain player $c$ receives from investing.

Given our assumption that the status quo outcome is no worse than the reunification outcome net of investments for player $s$, when $\frac{-\alpha - R_p^{SQ}}{1 -\de_p} > \frac{\beta+ R_c^{SQ}}{1 -\de_c}$, $p$ is willing to invest enough to create a large enough difference in $Q_{sn} - l_{sn}$ so that player $c$ will not find contesting the status quo to be in its interest. Whenever the status quo will be the outcome in this way, $c$ maximizes its utility by choice of $R_c^{SQ} =0$. Given that $R_p^{SQ}$ is non-negative, a sufficient condition to prevent $c$ from inducing the \emph{reunification} outcome is $\frac{-\alpha}{1 -\de_p}  > \frac{\beta}{1 -\de_c}$. \hfill $\blacksquare$
\\
\\
\noindent {\bf Proof of Lemma 3} \\
In the status quo, we have $U_{pn}^{SQ}= -R_p^{SQ}$ and $U_{cn}^{SQ}= -R_c^{SQ}$. If the outcome is recognition of the secessionists, per-period payoffs in this absorbing state become $U_{pn}^{REC}= \lambda -R_p^{REC}$ and $U_{cn}^{REC}= \nu -R_c^{REC}$ where $SQ$ and $REC$ distinguish investments under the status quo and recognition scenarios respectively.

Thus the difference in continuation values from switching from the status quo to recognition for player $p$ is $\frac{\lambda +R_p^{SQ}}{1 -\de_p} -R_p^{REC}$. For player $c$ it is $\frac{\nu + R_c^{SQ}}{1 -\de_c}-R_c^{REC}$. As noted above, $R_c^{REC} =0$. So $c$ is willing to invest up to $-\nu$ per period toward maintaining the status quo, and $p$ is willing to invest up to $\frac{\lambda +R_p^{SQ}}{1 -\de_p}$ toward recognition. We have shown that the largest possible investments by $p$ in the status quo are $\frac{\beta}{1 -\de_c}$ in $n=1$ and $\mu$ in each period thereafter, so the largest this difference can be for $p$ is $\frac{\lambda + \mu}{1 -\de_p} + \frac{\beta}{1 -\de_c}$.

Player $c$ would be willing to counter whatever investment player $p$ makes to try to induce recognition---e.g. counter an addition to $l_{gn}$ with an investment to $q_{gn}$---as long as $-\nu \geq L_{gn} - q_{gn}$, where $L_{gn}$ is gross of any investment by $p$.\footnote{As in the proof of Lemma 2, we have assumed that the payoff relationship at $n$ is such that ceding is better than fighting. If this relationship is reversed, similar analysis and results hold.} That is, the required investment must be no more than the loss player $c$ receives from investing.

Given our assumption that the status quo outcome is no worse than the recognition outcome net of investments for player $g$, $p$'s willingness to invest up to $\frac{\lambda + \mu}{1 -\de_p} + \frac{\beta}{1 -\de_c}$ can create a difference of at most that amount in $L_{gn} - q_{gn}$. 

Thus $\frac{-\nu}{1 -\de_c} > \frac{\lambda + \mu}{1 -\de_p} + \frac{\beta}{1 -\de_c}$ guarantees that player $c$ will always be willing to counter an investment by player $p$ toward achieving recognition, making $p$ unwilling to make that investment in the initial stage and making the response by player $c$ unnecessary. A slightly weaker version of this inequality is stated as condition 4 so that this deterrence dynamic is guaranteed to operate through the war lottery as well, as will be seen in Lemma 4. \hfill $\blacksquare$
\\
\\
\noindent {\bf Proof of Lemma 4} \\
The difference in continuation values from switching from the status quo to taking the war lottery for player $p$ is $\frac{\lambda p_1 + \alpha p_2 }{1 -\de_p} +R_p^{SQ} -R_p^{WAR}$. For player $c$ it is $\frac{\nu p_1 + \beta p_2}{1 -\de_c} + R_c^{SQ} -R_c^{WAR}$. As noted above, $R_c^{SQ} =0$. So $c$ is willing to invest up to $\frac{\nu p_1 + \beta p_2}{1 -\de_c}$ toward preventing fighting if this quantity is negative (leave the case where it is non-negative until later).

$p$ is willing to invest up to $\frac{\lambda p_1 + \alpha p_2 }{1 -\de_p} +R_p^{SQ}$ toward instigating fighting when this quantity is positive, and will not invest when the quantity is non-positive. Thus we must only examine the case where this quantity is positive.

We will again set $R_p^{SQ}$ at its largest value of $\frac{\mu}{1-\de_p} + \frac{\beta}{1 - \de_c}$ as this is most restrictive. In this only case of interest, we have $\frac{\alpha p_2}{1 -\de_p} +\frac{\lambda p_1}{1 -\de_p} + \frac{\mu}{1-\de_p} + \frac{\beta}{1 - \de_c}> 0$. Note that the first term is negative while the rest are positive.

Multiplying Condition 3 by $-p_{2s}$, we have $\frac{-\beta \cdot p_{2s}}{1-\de_c} \geq \frac{\alpha \cdot p_{2s}}{1-\de_p}$. Using this with Condition 4, we have
\[
  \textstyle -\left(\frac{\nu p_{1s} + \beta p_{2s}}{1 -\de_c}\right) = - \frac{\nu p_{1s}}{1 -\de_c} - \frac{\beta p_{2s}}{1 -\de_c} \geq \frac{\alpha p_{2s}}{1 -\de_p} +\frac{\lambda p_{1s}}{1 -\de_p} + \frac{\mu}{1-\de_p} + \frac{\beta}{1 - \de_c}> 0 \ \forall p_{1s},p_{2s}
\]

Hence when player $p$'s gain from instigating war is positive, player $c$'s loss is negative and larger in absolute magnitude. Player $c$ will thus counter any such investment by player $p$, deterring the investment in the first place. \hfill $\blacksquare$
\\

\end{document}