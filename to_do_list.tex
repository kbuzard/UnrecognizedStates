\documentclass[12pt]{article}

\addtolength{\textwidth}{1.4in}
\addtolength{\oddsidemargin}{-.7in} %left margin
\addtolength{\evensidemargin}{-.7in}
\setlength{\textheight}{8.5in}
\setlength{\topmargin}{0.0in}
\setlength{\headsep}{0.0in}
\setlength{\headheight}{0.0in}
\setlength{\footskip}{.5in}
\renewcommand{\baselinestretch}{1.0}
\setlength{\parindent}{0pt}
\linespread{1.1}

\usepackage{amssymb, amsmath, amsthm, bm}
\usepackage{graphicx,csquotes,verbatim}
\usepackage[backend=biber,block=space,style=authoryear]{biblatex}
\setlength{\bibitemsep}{\baselineskip}
\usepackage[american]{babel}
%dell laptop
\addbibresource{C:/Users/Kristy/Dropbox/Research/xBibs/tradeagreements.bib}
%\addbibresource{C:/Users/Kristy/Documents/Dropbox/Research/xBibs/tradeagreements.bib}
\renewcommand{\newunitpunct}{,}
\renewbibmacro{in:}{}


\DeclareMathOperator*{\argmax}{arg\,max}
\usepackage{xcolor}
\hbadness=10000

\newcommand{\ve}{\varepsilon}
\newcommand{\ov}{\overline}
\newcommand{\un}{\underline}
\newcommand{\ta}{\theta}
\newcommand{\al}{\alpha}
\newcommand{\Ta}{\Theta}
\newcommand{\expect}{\mathbb{E}}
\newcommand{\de}{\delta}

\begin{document}
\begin{center}
Unrecognized States To-do List (JLEO Revision)
\end{center}


\vskip.3in
Kristy
\begin{itemize}
	\item Recast Proposition 1 as a ``class of games'' instead of ``sufficient conditions'' (Reviewer 1)
		\begin{itemize}
			\item ``interpret these conditions as characterizations of “a class” of games, where this class would satisfy all the assumptions on payoffs (instead of saying sufficient conditions).
			\item I would recommend clearing up Section 3.1 in terms of packaging the results: 
					\begin{enumerate}
						\item Identify a class of games that satisfies the assumptions on payoffs (sufficient conditions (1) through (6) in Proposition 1). Denote this class as G for example. 
						\item Formally define the “status quo” equilibrium -- an equilibrium in which the equilibrium outcome is perpetual unrecognized statehood with all the actors’ strategies specified. 
						\item State a much clearer and simpler proposition, e.g., For any game G, there exists a status quo equilibrium.
						\item Then subsection 3.2 would basically be a discussion of this class of games. 
					\end{enumerate}
		\end{itemize}
	\item Clearly specify punishments / equilibrium path (editor)
		\begin{itemize}
			\item ``Clear and detailed description of the noncooperative game''
			\item ``I would also like to see the key logic laid out better in the text, and the full equilibrium strategy profiles noted so that it is clear how the players are rewarding and punishing each other on and off the equilibrium path''
		\end{itemize}
	\item Make sure equilibrium concept is clearly stated and defined
		\begin{itemize}
			\item ``Because the model is not a repeated game, due to the state variable just described, it may be a bit confusing to some readers, and so it is important to provide formal definitions and to clearly describe the equilibrium concept being utilized. It seems to me that, as Referee 2 states, you are using the Markov perfect equilibrium concept.''
		\end{itemize}
	\item Think through reviewer 2's suggestion about uncertainty [DONE]
		\begin{itemize}
			\item How would model actually work with asymmetric information? [NOT GOING TO DO]
		\end{itemize}
	\item Take indecisive war to decisive war [DONE]
	\item Change sign on payoff functions for $c$ and $p$ [DONE]
	\item Homogenize discount factors [DONE]
	\item State process governing status-quo payoffs clearly [DONE]
\end{itemize}


Ben
\begin{itemize}
	\item R1: Finish \& Double Check Fearon reference [DONE]
		\item add appropriate language to memo if necessary [DONE]

	\item R1: Talk to Kristy about whether we want to offer a numerical example a la what R1 requests. [DONE]
	
	\item R1: Tone down our claims about being novel and unique [DONE]
	
	\item R1: Double check that we have fixed the typo identified by R1 [DONE]
	
	\item R3: Read the literature referenced by R3 and add some cites [DONE]
	
	\item R3: Read Berg 2009 in particular and engage it at least briefly in the paper. [DONE]

\end{itemize}


\end{document}