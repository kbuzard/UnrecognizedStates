\documentclass[12pt]{article}

\addtolength{\textwidth}{1.4in}
\addtolength{\oddsidemargin}{-.7in} %left margin
\addtolength{\evensidemargin}{-.7in}
\setlength{\textheight}{8.5in}
\setlength{\topmargin}{0.0in}
\setlength{\headsep}{0.0in}
\setlength{\headheight}{0.0in}
\setlength{\footskip}{.5in}
\renewcommand{\baselinestretch}{1.0}
\setlength{\parindent}{0pt}
\linespread{1.1}

\usepackage{amssymb, amsmath, amsthm, bm}
\usepackage{graphicx,csquotes,verbatim}
\usepackage[backend=biber,block=space,style=authoryear]{biblatex}
\setlength{\bibitemsep}{\baselineskip}
\usepackage[american]{babel}
%dell laptop
\addbibresource{C:/Users/Kristy/Dropbox/Research/xBibs/tradeagreements.bib}
%\addbibresource{C:/Users/Kristy/Documents/Dropbox/Research/xBibs/tradeagreements.bib}
\renewcommand{\newunitpunct}{,}
\renewbibmacro{in:}{}


\DeclareMathOperator*{\argmax}{arg\,max}
\usepackage{xcolor}
\hbadness=10000

\newcommand{\ve}{\varepsilon}
\newcommand{\ov}{\overline}
\newcommand{\un}{\underline}
\newcommand{\ta}{\theta}
\newcommand{\al}{\alpha}
\newcommand{\Ta}{\Theta}
\newcommand{\expect}{\mathbb{E}}
\newcommand{\de}{\delta}

\begin{document}
\begin{center}
Equilibrium Strategy Profiles (For JLEO Revision)
\end{center}


\vskip.3in
First, need to get timing clear for evolution of state variable. Take state variable to be $q_{st}$.
\begin{itemize}
	\item $q_{s1}$: beginning of the world (or, $Q_{s0} - \mu = q_{s1}$)
	\item $Q_{s1} = q_{s1} + R_{p1} + R_{c1}$
		\begin{itemize}
			\item $q_{s2} = Q_{s1} - \mu$
		\end{itemize}
\end{itemize}

\vskip.2in
Strategies (this is for the most interesting state variable, but could write it out for the other five)
\begin{itemize}
	\item For patron: function of $q_{st}$
	\item For c: function of $q_{st} + R_{pt}$
	\item For s: function of $Q_{st}=q_{st} + R_{pt} + R_{ct}$
	\item For g: not a function of $q_{st}$ at all
\end{itemize}

\vskip.5in	
According to Mailath and Samuelson (2006) p. 177.
\begin{itemize}
	\item The strategy profile $\sigma$ is a stationary Markov strategy if for any two ex post histories $\tilde{h}^t$ and $\tilde{h}^\tau$ (of equal or different lengths) terminating in the same state, $\sigma\left(\tilde{h}^t\right) = \sigma\left(\tilde{h}^\tau\right)$.
	\item The strategy profile $\sigma$ is a stationary Markov equilibrium if $\sigma$ is a stationary Markov strategy profile and a subgame-perfect equilibrium.
\end{itemize}

\vskip.5in
We have six state variables, so $s = \left(q_s,q_g,l_s,l_g,w_s,w_g \right)$.
\begin{itemize}
	\item Strategies are a function of all six state variables
		\begin{itemize}
			\item We can restrict attention to relevant ranges of the state variables. CAN I CHANGE PROP 1 PARTS 1 AND 2 TO BE INITIAL VALUES OF STATE VARIABLES? If so, then these conditions and what is necessary in each period to make the game continue.
		\end{itemize}
	\item Strategies for the patron and $c$ are also how much to invest in each of the six state variables. Some can be ruled out by preference assumptions:
		\begin{itemize}
			\item $c$ dislikes war, so will never invest in $w_s$ or $w_g$. It would also not want to make the government lose, so won't invest in $l_g$ either.
			\item Because the patron's preferences are aligned with the secessionists and against the government, it never invests in $w_g$ or $l_s$.
		\end{itemize}
\end{itemize}

\vskip.5in
This leaves four state varibles in which the patron may invest: $q_s$, $q_g$, $l_g$ and $w_s$. Three in which $c$ may invest: $q_s$, $q_g$, $l_s$.
\begin{itemize}
	\item Also, Gov't / Secessionists: Choose unilateral, simultaneous best responses depending on magnitudes of $Q_{i1}$, $L_{i1}$ and $\omega_{i1}$
		\begin{itemize}
			\item Game only continues if (SQ,SQ) or (Cede, Cede) was played
		\end{itemize}
\end{itemize}

\vskip.2in
Assumptions
\begin{itemize}
	\item $Q_{s0} \geq L_{s0} \Rightarrow q_{s1} + \mu \geq l_{s1}$
	\item $Q_{g0} \geq L_{g0} \Rightarrow q_{g1} \geq l_{g1}$
\end{itemize}

\vskip.2in
Period 1
\begin{enumerate}
	\item Patron would like $Q_{s1} \geq L_{s1}$ to prevent $c$ from incentivizing secessionists to Cede
		\begin{itemize}
			\item Equivalent to $q_{s1} + R_{p1} \geq l_{s1} + R_{c1}$ / $Q_{s0} + \mu + R_{p1} \geq L_{s0} + R_{c1}$ / $R_{p1} \geq R_{c1} - \left( Q_{s0} - \mu - L_{s0}\right)$
				\begin{itemize}
					\item $R_{c1} \leq \frac{\beta}{1-\de} \Rightarrow$ $R_{p1} \geq \frac{\beta}{1-\de} - \left( Q_{s0} - \mu - L_{s0}\right)$ allows the patron to ensure the original inequality
					\item Initial assumption means $R_{p1} \geq \frac{\beta}{1-\de} + \mu$ is a tighter condition
					\item Since patron is willing to pay up to $R_{p1} = \frac{\alpha}{1-\de}$, when $\frac{\alpha}{1-\de} \geq \frac{\beta}{1-\de} + \mu$, it will invest $R_{p1} = \frac{\beta}{1-\de} - \left( Q_{s0} - \mu - L_{s0}\right)$ to augment $q_{s1}$ if this is greater than 0. Else, $R_{p1} = 0$. 
					  \begin{enumerate}
							\item If $R_{p1} + \left( Q_{s0} - \mu - L_{s0}\right) \geq \frac{\beta}{1-\de}$, $R_{c1} =0$. (SQ,SQ) is played and game continues.
							\item If instead $R_{p1} + \left( Q_{s0} - \mu - L_{s0}\right) < \frac{\beta}{1-\de}$ (i.e. assumption doesn't hold), $R_{c1} = l_{s1} - \left(q_{s1} + R_{p1} \right) + \ve$ to augment $l_{s1}$. Note optimal $R_{p1}$ in this case is 0.
								\begin{itemize}
									\item (SQ,Cede) is played and game ends
								\end{itemize}
						\end{enumerate}
				\end{itemize}
		\end{itemize}	
	\item Patron would like $L_{g1} > Q_{g1}$ to achieve recognition directly. Must invest to augment $l_{g1}$
		\begin{itemize}
			\item Equivalent to $l_{g1} + R_{p1} > q_{g1} + R_{c1}$ / $L_{g0} + R_{p1} > Q_{g0} + R_{c1}$ / $R_{p1} > R_{c1} + \left( Q_{g0} - L_{g0}\right)$
				\begin{itemize}
					\item $R_{c1} \leq \frac{\nu}{1-\de} \Rightarrow$ $R_{p1} > \frac{\nu}{1-\de} + \left( Q_{g0} - L_{g0}\right)$ allows the patron to ensure the original inequality
					\item Since patron is willing to pay up to $R_{p1} = \frac{\lambda +\mu + \beta}{1-\de}$, when $\frac{\lambda +\mu + \beta}{1-\de} \leq \frac{\nu}{1-\de}$, patron will invest 0 in $l_{g1}$. $R_{c1} = 0$ as well.
					\item If instead assumption 4 did not hold and $R_{p1} > \frac{\nu}{1-\de} + \left( Q_{g0} - L_{g0}\right)$, it will invest this amount to augment $l_{g1}$. Again, $R_{c1} = 0$. (Cede, SQ) is played and game ends.
				\end{itemize}
		\end{itemize}
	\item Patron might want to instigate fighting to have a $p_{1s}$ probability of achieving recognition indirectly. If $\frac{\lambda p_{1s} - \alpha (1-p_{1s}) + \mu + \beta}{1 - \de}$ is positive, invest in $w_{s1}$ to change $\omega_{s1}$. Will not invest when the quantity is non-positive. Inequality is $\omega_{s1} > Q_{s1}$
		\begin{itemize}
			\item $\omega_{s1} = -\zeta_{s}(1-\de) + W_{s1}p_{1s} + L_{s1}(1-p_{1s}) > Q_{s1}$
			\item $-\zeta_{s}(1-\de) + (W_{s0}+R_{p1})p_{1s} + L_{s0}(1-p_{1s}) > Q_{s0} - \mu + R_{c1}$
			\item $R_{p1}p_{1s} > R_{c1} - \mu + Q_{s0} - \left(-\zeta_{s}(1-\de) + W_{s0}p_{1s} + L_{s0}(1-p_{1s})\right)$
			\item $R_{c1} \leq \frac{\nu p_{1s} - \beta (1-p_{1s})}{1 -\de}$, so if $R_{p1}p_{1s} > \frac{\nu p_{1s} - \beta (1-p_{1s})}{1 -\de} - \mu + Q_{s0} - \left(-\zeta_{s}(1-\de) + W_{s0}p_{1s} + L_{s0}(1-p_{1s})\right)$, $R_{c1}=0$ and (SQ,Fight) is played in third stage. Outcome depends on war lottery.
			\item Patron will spend at most $\frac{\lambda p_{1s} - \alpha (1-p_{1s}) +\mu + \beta}{1 -\de} - \mu$
				\begin{itemize}
					\item So need
						\begin{multline}
							\textstyle p_{1s}\left[\frac{\lambda p_{1s} - \alpha (1-p_{1s}) +\mu + \beta}{1 -\de} - \mu\right] < \frac{\lambda p_{1s} - \alpha (1-p_{1s}) +\mu + \beta}{1 -\de} - \mu \leq \frac{\nu p_{1s} - \beta (1-p_{1s})}{1 -\de} - \mu \\ \leq \frac{\nu p_{1s} - \beta (1-p_{1s})}{1 -\de} - \mu + Q_{s0} - \left(-\zeta_{s}(1-\de) + W_{s0}p_{1s} + L_{s0}(1-p_{1s})\right)
						\end{multline}
				\end{itemize}
		\end{itemize}
	\item Patron could also invest in $q_g$ to counter investment by $c$ in government's payoffs. But this would only been needed if $c$ had incentive to invest in $g$ playing Fight, and we've ruled that out.
\end{enumerate}


\vskip.2in
Now generalize profiles for all $t$
\begin{enumerate}
	\item Patron would like $Q_{st} \geq L_{st}$ to prevent $c$ from incentivizing secessionists to Cede
		\begin{itemize}
			\item Equivalent to $q_{st} + R_{pt} \geq l_{st} + R_{ct}$ / $Q_{s,t-1} + \mu + R_{pt} \geq L_{s,t-1} + R_{ct}$ / $R_{pt} \geq R_{ct} - \left( Q_{s,t-1} - \mu - L_{s,t-1}\right)$
				\begin{itemize}
					\item $R_{ct} \leq \frac{\beta}{1-\de} \Rightarrow$ $R_{pt} \geq \frac{\beta}{1-\de} - \left( Q_{s,t-1} - \mu - L_{s,t-1}\right)$ allows the patron to ensure the original inequality
					\item Assumption 1 combined with equilibrium play (equivalently, if $L_{s,t-1} > Q_{s,t-1}$, the game would have ended before the start of period $t$) means $R_{pt} \geq \frac{\beta}{1-\de} + \mu$ is a tighter condition
					\item Since patron is willing to pay up to $R_{pt} = \frac{\alpha}{1-\de}$, when $\frac{\alpha}{1-\de} \geq \frac{\beta}{1-\de} + \mu$, it will invest $R_{pt} = \frac{\beta}{1-\de} - \left( Q_{s,t-1} - \mu - L_{s,t-1}\right)$ to augment $q_{st}$ if this is greater than 0. Else, $R_{pt} = 0$. 
					  \begin{enumerate}
							\item If $R_{pt} + \left( Q_{s,t-1} - \mu - L_{s,t-1}\right) \geq \frac{\beta}{1-\de}$, $R_{ct} =0$. (SQ,SQ) is played and game continues.
							\item If instead $R_{pt} + \left( Q_{s,t-1} - \mu - L_{s,t-1}\right) < \frac{\beta}{1-\de}$ (i.e. assumption 3 doesn't hold), $R_{ct} = l_{st} - \left(q_{st} + R_{pt} \right) + \ve$ where $\ve$ is small to augment $l_{st}$. Note optimal $R_{pt}$ in this case is 0.
								\begin{itemize}
									\item (SQ,Cede) is played and game ends
								\end{itemize}
						\end{enumerate}
				\end{itemize}
		\end{itemize}	
	\item Patron would like $L_{gt} > Q_{gt}$ to achieve recognition directly. Must invest to augment $l_{gt}$
		\begin{itemize}
			\item Equivalent to $l_{gt} + R_{pt} > q_{gt} + R_{ct}$ / $L_{g,t-1} + R_{pt} > Q_{g,t-1} + R_{ct}$ / $R_{pt} > R_{ct} + \left( Q_{g,t-1} - L_{g,t-1}\right)$
				\begin{itemize}
					\item $R_{ct} \leq \frac{\nu}{1-\de} \Rightarrow$ $R_{pt} > \frac{\nu}{1-\de} + \left( Q_{g,t-1} - L_{g,t-1}\right)$ allows the patron to ensure the original inequality
					\item Since patron is willing to pay up to $R_{pt} = \frac{\lambda +\mu + \beta}{1-\de} - \mu$, when $\frac{\lambda +\mu + \beta}{1-\de} -\mu \leq \frac{\nu}{1-\de}$, patron will invest 0 in $l_{gt}$. $R_{ct} = 0$ as well.
					\item If instead assumption 4 did not hold and $R_{pt} > \frac{\nu}{1-\de} + \left( Q_{g,t-1} - L_{g,t-1}\right)$, patron will invest this amount to augment $l_{gt}$. Again, $R_{ct} = 0$. (Cede, SQ) is played and game ends.
				\end{itemize}
		\end{itemize}
	\item Patron might want to instigate fighting to have a $p_{1s}$ probability of achieving recognition indirectly. If $\frac{\lambda p_{1s} - \alpha (1-p_{1s}) + \mu + \beta}{1 - \de} -\mu$ is positive, invest in $w_{st}$ to change $\omega_{st}$. Will not invest when the quantity is non-positive. Inequality is $\omega_{st} > Q_{st}$
		\begin{itemize}
			\item $\omega_{st} = -\zeta_{s}(1-\de) + W_{st}p_{1s} + L_{st}(1-p_{1s}) > Q_{st}$
			\item $-\zeta_{s}(1-\de) + (W_{s,t-1}+R_{pt})p_{1s} + L_{s,t-1}(1-p_{1s}) > Q_{s,t-1} - \mu + R_{ct}$
			\item $R_{pt}p_{1s} > R_{ct} - \mu + Q_{s,t-1} - \left(-\zeta_{s}(1-\de) + W_{s,t-1}p_{1s} + L_{s,t-1}(1-p_{1s})\right)$
			\item $R_{ct} \leq \frac{\nu p_{1s} - \beta (1-p_{1s})}{1 -\de}$, so if $R_{pt}p_{1s} > \frac{\nu p_{1s} - \beta (1-p_{1s})}{1 -\de} - \mu + Q_{s,t-1} - \left(-\zeta_{s}(1-\de) + W_{s,t-1}p_{1s} + L_{s,t-1}(1-p_{1s})\right)$, $R_{ct}=0$ and (SQ,Fight) is played in third stage. Outcome depends on war lottery.
			\item Patron will spend at most $\frac{\lambda p_{1s} - \alpha (1-p_{1s}) +\mu + \beta}{1 -\de} - \mu$
				\begin{itemize}
					\item So need
						\begin{multline}
							\textstyle p_{1s}\left[\frac{\lambda p_{1s} - \alpha (1-p_{1s}) +\mu + \beta}{1 -\de} - \mu\right] < \frac{\lambda p_{1s} - \alpha (1-p_{1s}) +\mu + \beta}{1 -\de} - \mu \leq \frac{\nu p_{1s} - \beta (1-p_{1s})}{1 -\de} - \mu \\ \leq \frac{\nu p_{1s} - \beta (1-p_{1s})}{1 -\de} - \mu + Q_{s,t-1} - \left(-\zeta_{s}(1-\de) + W_{s,t-1}p_{1s} + L_{s,t-1}(1-p_{1s})\right)
						\end{multline}
				\end{itemize}
		\end{itemize}
	\item Patron could also invest in $q_g$ to counter investment by $c$ in government's payoffs. But this would only been needed if $c$ had incentive to invest in $g$ playing Fight, and we've ruled that out.
\end{enumerate}


\newpage
Prose version of equilibrium strategy profiles: \\

On the equilibrium path, the patron invests to avoid the reunification outcome. That is, it invests in the secessionists' status quo payoffs to deter the international community from investing in the secessionists' payoffs from ceding. In period $t$, the patron invests $R_{pt} = \max\left\{\frac{\beta}{1-\de} - \left( Q_{s,t-1} - \mu - L_{s,t-1}\right),0\right\}$ to augment the status quo payoffs of the secessionists, $q_{st}$. Player $c$ then chooses $R_{ct} =0$ so that playing status quo is a best response for both inside players. (SQ,SQ) is played in the third stage and the game continues to period $t+1$.

If instead $R_{pt} < \frac{\beta}{1-\de} - \left( Q_{s,t-1} - \mu - L_{s,t-1}\right)$ (i.e. assumption 3 doesn't hold), player $c$ will invest $R_{ct} = l_{st} - \left(q_{st} + R_{pt} \right) + \ve$, for $\ve$ small, to augment $l_{st}$. In this case, status quo remains a best response for the government but cede becomes the best response for the secessionists. The game ends with the secessionist territory being reunited. This path is ruled out by Assumption 3 (see Lemma 2).

The patron would prefer to invest to provoke the recognition outcome, which it can attempt to do directly by investing in the government's payoffs from ceding, $l_{gt}$. If the patron invests $R_{pt} > \frac{\nu}{1-\de} + \left( Q_{g,t-1} - L_{g,t-1}\right)$ to augment $l_{gt}$, player $c$ will then choose $R_{ct} = 0$ and cede will be the best response for the government. Status quo remains the best response for the secessionists, so the game ends with the secessionists gaining recognition. This path is ruled out by Assumption 4 (see Lemma 3).

As long as $R_{pt} \leq \frac{\nu}{1-\de} + \left( Q_{g,t-1} - L_{g,t-1}\right)$, player $c$ will counter with $R_{ct} = R_{pt} - \left( Q_{g,t-1} - L_{g,t-1}\right)$ to augment $l_{gt}$. This makes status quo a best response for the government as well as for the secessionists so that the status quo equilibrium is played and the game continues to period $t+1$. Note that the optimal investments in the government's payoffs zero.

Instead of instigating recognition directly by improving the government's payoffs from ceding, the patron might want to instigate fighting, which would lead to recognition with probability $p_{1s}$. If the patron invests $R_{pt}p_{1s} > \frac{\nu p_{1s} - \beta (1-p_{1s})}{1 -\de} - \mu + Q_{s,t-1} - \left(-\zeta_{s}(1-\de) + W_{s,t-1}p_{1s} + L_{s,t-1}(1-p_{1s})\right)$ to augment $w_{s,t}$, player $c$ will choose $R_{ct}=0$ and Fight becomes a best response for the secessionists so (SQ,Fight) is played in third stage. The game will end with an outcome that depends on the war lottery. This path is ruled out by a combination of Assumptions 3 and 4 (see Lemma 4).

If the patron spends less than this amount, player $c$ will counter by investing $R_{ct} = R_{pt}p_{1s} + \mu - Q_{s,t-1} + \left(-\zeta_{s}(1-\de) + W_{s,t-1}p_{1s} + L_{s,t-1}(1-p_{1s})\right)$ in $q_{st}$, the secessionists' status quo payoffs. This makes playing Status Quo a best response for both players so that the status quo equilibrium is played and the game continues to period $t+1$. Here again the optimal investment by both parties is zero.\footnote{Note that the patron could also invest in $q_{gt}$ to counter investment by player $c$ in the government's payoffs from war. But this would only been needed if $c$ had incentive to invest in $g$ playing Fight, and we've ruled that out.}

\end{document}