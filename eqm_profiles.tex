\documentclass[12pt]{article}

\addtolength{\textwidth}{1.4in}
\addtolength{\oddsidemargin}{-.7in} %left margin
\addtolength{\evensidemargin}{-.7in}
\setlength{\textheight}{8.5in}
\setlength{\topmargin}{0.0in}
\setlength{\headsep}{0.0in}
\setlength{\headheight}{0.0in}
\setlength{\footskip}{.5in}
\renewcommand{\baselinestretch}{1.0}
\setlength{\parindent}{0pt}
\linespread{1.1}

\usepackage{amssymb, amsmath, amsthm, bm}
\usepackage{graphicx,csquotes,verbatim}
\usepackage[backend=biber,block=space,style=authoryear]{biblatex}
\setlength{\bibitemsep}{\baselineskip}
\usepackage[american]{babel}
%dell laptop
\addbibresource{C:/Users/Kristy/Dropbox/Research/xBibs/tradeagreements.bib}
%\addbibresource{C:/Users/Kristy/Documents/Dropbox/Research/xBibs/tradeagreements.bib}
\renewcommand{\newunitpunct}{,}
\renewbibmacro{in:}{}


\DeclareMathOperator*{\argmax}{arg\,max}
\usepackage{xcolor}
\hbadness=10000

\newcommand{\ve}{\varepsilon}
\newcommand{\ov}{\overline}
\newcommand{\un}{\underline}
\newcommand{\ta}{\theta}
\newcommand{\al}{\alpha}
\newcommand{\Ta}{\Theta}
\newcommand{\expect}{\mathbb{E}}
\newcommand{\de}{\delta}

\begin{document}
\begin{center}
Equilibrium Strategy Profiles (For JLEO Revision)
\end{center}


\vskip.3in
First, need to get timing clear for evolution of state variable. Take state variable to be $q_{st}$.
\begin{itemize}
	\item $q_{s1}$: beginning of the world (or, $Q_{s0} - \mu = q_{s1}$)
	\item $Q_{s1} = q_{s1} + R_{p1} + R_{c1}$
		\begin{itemize}
			\item $q_{s2} = Q_{s1} - \mu$
		\end{itemize}
\end{itemize}

\vskip.2in
Strategies (this is for the most interesting state variable, but could write it out for the other five)
\begin{itemize}
	\item For patron: function of $q_{st}$
	\item For c: function of $q_{st} + R_{pt}$
	\item For s: function of $Q_{st}=q_{st} + R_{pt} + R_{ct}$
	\item For g: not a function of $q_{st}$ at all
\end{itemize}

\vskip.5in	
According to Mailath and Samuelson (2006) p. 177.
\begin{itemize}
	\item The strategy profile $\sigma$ is a stationary Markov strategy if for any two ex post histories $\tilde{h}^t$ and $\tilde{h}^\tau$ (of equal or different lengths) terminating in the same state, $\sigma\left(\tilde{h}^t\right) = \sigma\left(\tilde{h}^\tau\right)$.
	\item The strategy profile $\sigma$ is a stationary Markov equilibrium if $\sigma$ is a stationary Markov strategy profile and a subgame-perfect equilibrium.
\end{itemize}

\vskip.5in
We have six state variables, so $s = \left(q_s,q_g,l_s,l_g,w_s,w_g \right)$.
\begin{itemize}
	\item Strategies are a function of all six state variables
		\begin{itemize}
			\item We can restrict attention to relevant ranges of the state variables. CAN I CHANGE PROP 1 PARTS 1 AND 2 TO BE INITIAL VALUES OF STATE VARIABLES? If so, then these conditions and what is necessary in each period to make the game continue.
		\end{itemize}
	\item Strategies for the patron and $c$ are also how much to invest in each of the six state variables. Some can be ruled out by preference assumptions:
		\begin{itemize}
			\item $c$ dislikes war, so will never invest in $w_s$ or $w_g$. It would also not want to make the government lose, so won't invest in $l_g$ either.
			\item Because the patron's preferences are aligned with the secessionists and against the government, it never invests in $w_g$ or $l_s$.
		\end{itemize}
\end{itemize}

\vskip.5in
This leaves four state varibles in which the patron may invest: $q_s$, $q_g$, $l_g$ and $w_s$. Three in which $c$ may invest: $q_s$, $q_g$, $l_s$.
\begin{itemize}
	\item Also, Gov't / Secessionists: Choose unilateral, simultaneous best responses depending on magnitudes of $Q_{i1}$, $L_{i1}$ and $\omega_{i1}$
		\begin{itemize}
			\item Game only continues if (SQ,SQ) or (Cede, Cede) was played
		\end{itemize}
\end{itemize}

\vskip.2in
Assumptions
\begin{itemize}
	\item $Q_{s0} \geq L_{s0} \Rightarrow q_{s1} + \mu \geq l_{s1}$
	\item $Q_{g0} \geq L_{g0} \Rightarrow q_{g1} \geq l_{g1}$
\end{itemize}

\vskip.2in
Period 1
\begin{enumerate}
	\item Patron would like $Q_{s1} \geq L_{s1}$ to prevent $c$ from incentivizing secessionists to Cede
		\begin{itemize}
			\item Equivalent to $q_{s1} + R_{p1} \geq l_{s1} + R_{c1}$ / $Q_{s0} + \mu + R_{p1} \geq L_{s0} + R_{c1}$ / $R_{p1} \geq R_{c1} - \left( Q_{s0} - \mu - L_{s0}\right)$
				\begin{itemize}
					\item $R_{c1} \leq \frac{\beta}{1-\de} \Rightarrow$ $R_{p1} \geq \frac{\beta}{1-\de} - \left( Q_{s0} - \mu - L_{s0}\right)$ allows the patron to ensure the original inequality
					\item Initial assumption means $R_{p1} \geq \frac{\beta}{1-\de} + \mu$ is a tighter condition
					\item Since patron is willing to pay up to $R_{p1} = \frac{\alpha}{1-\de}$, when $\frac{\alpha}{1-\de} \geq \frac{\beta}{1-\de} + \mu$, it will invest $R_{p1} = \frac{\beta}{1-\de} - \left( Q_{s0} - \mu - L_{s0}\right)$ to augment $q_{s1}$ if this is greater than 0. Else, $R_{p1} = 0$. 
					  \begin{enumerate}
							\item If $R_{p1} + \left( Q_{s0} - \mu - L_{s0}\right) \geq \frac{\beta}{1-\de}$, $R_{c1} =0$. (SQ,SQ) is played and game continues.
							\item If instead $R_{p1} + \left( Q_{s0} - \mu - L_{s0}\right) < \frac{\beta}{1-\de}$ (i.e. assumption doesn't hold), $R_{c1} = l_{s1} - \left(q_{s1} + R_{p1} \right) + \ve$ to augment $l_{s1}$. Note optimal $R_{p1}$ in this case is 0.
								\begin{itemize}
									\item (SQ,Cede) is played and game ends
								\end{itemize}
						\end{enumerate}
				\end{itemize}
		\end{itemize}	
	\item Patron would like $L_{g1} > Q_{g1}$ to achieve recognition directly. Must invest to augment $l_{g1}$
		\begin{itemize}
			\item Equivalent to $l_{g1} + R_{p1} > q_{g1} + R_{c1}$ / $L_{g0} + R_{p1} > Q_{g0} + R_{c1}$ / $R_{p1} > R_{c1} + \left( Q_{g0} - L_{g0}\right)$
				\begin{itemize}
					\item $R_{c1} \leq \frac{\nu}{1-\de} \Rightarrow$ $R_{p1} > \frac{\nu}{1-\de} + \left( Q_{g0} - L_{g0}\right)$ allows the patron to ensure the original inequality
					\item Since patron is willing to pay up to $R_{p1} = \frac{\lambda +\mu + \beta}{1-\de}$, when $\frac{\lambda +\mu + \beta}{1-\de} \leq \frac{\nu}{1-\de}$, patron will invest 0 in $l_{g1}$. $R_{c1} = 0$ as well.
					\item If instead assumption 4 did not hold and $R_{p1} > \frac{\nu}{1-\de} + \left( Q_{g0} - L_{g0}\right)$, it will invest this amount to augment $l_{g1}$. Again, $R_{c1} = 0$. (Cede, SQ) is played and game ends.
				\end{itemize}
		\end{itemize}
	\item Patron might also want to instigate fighting. If $\frac{\lambda p_{1s} - \alpha (1-p_{1s}) + \mu + \beta}{1 - \de}$ is positive, invest in $w_{s1}$ to change $\omega_{s1}$. Will not invest in $w_{s1}$ when the quantity is non-positive.
		\begin{itemize}
			\item $c$ will counter up to $\frac{\nu p_{1s} - \beta (1-p_{1s})}{1 -\de}$
		\end{itemize}
	\item Patron could also invest in $q_g$ to counter investment by $c$ in government's payoffs. But this would only been needed if $c$ had incentive to invest in $g$ playing Fight, and we've ruled that out.
\end{enumerate}

\end{document}